% Options for packages loaded elsewhere
\PassOptionsToPackage{unicode}{hyperref}
\PassOptionsToPackage{hyphens}{url}
\PassOptionsToPackage{dvipsnames,svgnames,x11names}{xcolor}
%
\documentclass[
  authoryear,
  preprint,
  3p]{elsarticle}

\usepackage{amsmath,amssymb}
\usepackage{iftex}
\ifPDFTeX
  \usepackage[T1]{fontenc}
  \usepackage[utf8]{inputenc}
  \usepackage{textcomp} % provide euro and other symbols
\else % if luatex or xetex
  \usepackage{unicode-math}
  \defaultfontfeatures{Scale=MatchLowercase}
  \defaultfontfeatures[\rmfamily]{Ligatures=TeX,Scale=1}
\fi
\usepackage{lmodern}
\ifPDFTeX\else  
    % xetex/luatex font selection
\fi
% Use upquote if available, for straight quotes in verbatim environments
\IfFileExists{upquote.sty}{\usepackage{upquote}}{}
\IfFileExists{microtype.sty}{% use microtype if available
  \usepackage[]{microtype}
  \UseMicrotypeSet[protrusion]{basicmath} % disable protrusion for tt fonts
}{}
\makeatletter
\@ifundefined{KOMAClassName}{% if non-KOMA class
  \IfFileExists{parskip.sty}{%
    \usepackage{parskip}
  }{% else
    \setlength{\parindent}{0pt}
    \setlength{\parskip}{6pt plus 2pt minus 1pt}}
}{% if KOMA class
  \KOMAoptions{parskip=half}}
\makeatother
\usepackage{xcolor}
\setlength{\emergencystretch}{3em} % prevent overfull lines
\setcounter{secnumdepth}{5}
% Make \paragraph and \subparagraph free-standing
\ifx\paragraph\undefined\else
  \let\oldparagraph\paragraph
  \renewcommand{\paragraph}[1]{\oldparagraph{#1}\mbox{}}
\fi
\ifx\subparagraph\undefined\else
  \let\oldsubparagraph\subparagraph
  \renewcommand{\subparagraph}[1]{\oldsubparagraph{#1}\mbox{}}
\fi


\providecommand{\tightlist}{%
  \setlength{\itemsep}{0pt}\setlength{\parskip}{0pt}}\usepackage{longtable,booktabs,array}
\usepackage{calc} % for calculating minipage widths
% Correct order of tables after \paragraph or \subparagraph
\usepackage{etoolbox}
\makeatletter
\patchcmd\longtable{\par}{\if@noskipsec\mbox{}\fi\par}{}{}
\makeatother
% Allow footnotes in longtable head/foot
\IfFileExists{footnotehyper.sty}{\usepackage{footnotehyper}}{\usepackage{footnote}}
\makesavenoteenv{longtable}
\usepackage{graphicx}
\makeatletter
\def\maxwidth{\ifdim\Gin@nat@width>\linewidth\linewidth\else\Gin@nat@width\fi}
\def\maxheight{\ifdim\Gin@nat@height>\textheight\textheight\else\Gin@nat@height\fi}
\makeatother
% Scale images if necessary, so that they will not overflow the page
% margins by default, and it is still possible to overwrite the defaults
% using explicit options in \includegraphics[width, height, ...]{}
\setkeys{Gin}{width=\maxwidth,height=\maxheight,keepaspectratio}
% Set default figure placement to htbp
\makeatletter
\def\fps@figure{htbp}
\makeatother

\usepackage{fontspec}
\usepackage{multirow}
\usepackage{multicol}
\usepackage{colortbl}
\usepackage{hhline}
\newlength\Oldarrayrulewidth
\newlength\Oldtabcolsep
\usepackage{longtable}
\usepackage{array}
\usepackage{hyperref}
\usepackage{float}
\usepackage{wrapfig}
\usepackage{graphicx}
\usepackage{unicode-math}
\usepackage{hyperref}
\def\tightlist{}
\usepackage{setspace}
\doublespacing
\usepackage{lineno}
\linenumbers
\makeatletter
\@ifpackageloaded{caption}{}{\usepackage{caption}}
\AtBeginDocument{%
\ifdefined\contentsname
  \renewcommand*\contentsname{Table of contents}
\else
  \newcommand\contentsname{Table of contents}
\fi
\ifdefined\listfigurename
  \renewcommand*\listfigurename{List of Figures}
\else
  \newcommand\listfigurename{List of Figures}
\fi
\ifdefined\listtablename
  \renewcommand*\listtablename{List of Tables}
\else
  \newcommand\listtablename{List of Tables}
\fi
\ifdefined\figurename
  \renewcommand*\figurename{Figure}
\else
  \newcommand\figurename{Figure}
\fi
\ifdefined\tablename
  \renewcommand*\tablename{Table}
\else
  \newcommand\tablename{Table}
\fi
}
\@ifpackageloaded{float}{}{\usepackage{float}}
\floatstyle{ruled}
\@ifundefined{c@chapter}{\newfloat{codelisting}{h}{lop}}{\newfloat{codelisting}{h}{lop}[chapter]}
\floatname{codelisting}{Listing}
\newcommand*\listoflistings{\listof{codelisting}{List of Listings}}
\makeatother
\makeatletter
\makeatother
\makeatletter
\@ifpackageloaded{caption}{}{\usepackage{caption}}
\@ifpackageloaded{subcaption}{}{\usepackage{subcaption}}
\makeatother
\journal{Journal of Transport Geography}
\ifLuaTeX
  \usepackage{selnolig}  % disable illegal ligatures
\fi
\usepackage[]{natbib}
\bibliographystyle{elsarticle-harv}
\usepackage{bookmark}

\IfFileExists{xurl.sty}{\usepackage{xurl}}{} % add URL line breaks if available
\urlstyle{same} % disable monospaced font for URLs
\hypersetup{
  pdftitle={Exploring mobility of care with measures of accessibility},
  pdfauthor={AAA; BBB; CCC},
  pdfkeywords={Accessibility, Mobility of Care, Gender, cumulative
opportunities, Spatial Availability},
  colorlinks=true,
  linkcolor={blue},
  filecolor={Maroon},
  citecolor={Blue},
  urlcolor={Blue},
  pdfcreator={LaTeX via pandoc}}

\setlength{\parindent}{6pt}
\begin{document}

\begin{frontmatter}
\title{Exploring mobility of care with measures of accessibility}
\author[]{AAA%
\corref{cor1}%
}
 \ead{AAA@AAA} 
\author[]{BBB%
%
}
 \ead{BBB@BBB} 
\author[]{CCC%
%
}
 \ead{CCC@CCC} 


\cortext[cor1]{Corresponding author}



        
\begin{abstract}
Accessibility, the ease of interacting with potential opportunities, is
an increasingly important tool amongst transport planners aiming to
foster equitable and sustainable cities. However, in accessibility
research there is a historical focus on employment destinations that is
shaped by a masculinist transportation planning tradition. This paper
aims to counter this gendered bias by connecting the Mobility of Care
framework, a gender-aware transport planning conceptualisation to an
empirical accessibility analysis of care destinations in the City of
Hamilton, Canada. Care destinations are all the places one must visit to
sustain household needs such shopping, errands, and caring for others
(children and other dependents). Through the creation of a novel care
destination dataset, this paper considers access to care across
different modes of transport at two travel time thresholds (trips
shorter than 15-minutes and 30-minutes). The methods include using a
routinely used accessibility measure (cumulative opportunities) and a
novel competitive and singly-constrained accessibility measure (spatial
availability). Results indicate that accessibility to care destinations
by car is exceptionally high, and access by public transit, cycling and
by foot is low across the city with some exceptions in the inner city.
Notably, there are distinctions between both methods: cumulative
opportunities illustrate a more optimistic potential interaction
landscape for non-car modes, while the spatial availability measure
demonstrates a theoretically more realistic spatial distribution of care
destination availability of potential interaction. Neighbourhoods with
both low spatial availability to care and a high proportion of
low-income households are also identified as areas in need of
intervention. The manuscript and analysis is computationally
reproducible and openly available. The analysis presented demonstrates
methods planners can use to apply a gender-aware lens to accessibility
analysis. Further, results can inform policies aiming to encourage
sustainable mobility.
\end{abstract}





\begin{keyword}
    Accessibility \sep Mobility of Care \sep Gender \sep cumulative
opportunities \sep 
    Spatial Availability
\end{keyword}
\end{frontmatter}
    
\section{Introduction}\label{introduction}

A gender bias exists in transport research and policy
\citep{sanchezdemadariagaMobilityCareIntroducing2013, lawWomenTransportNew1999, siemiatyckiGenderedProductionInfrastructure2020}.
The field has historically focused predominately on the on-peak commute
to work. While most women participate in the labour force, the commute
is still a travel pattern more frequent among men
\citep{sanchezdemadariagaMobilityCareIntroducing2013}. Women, on the
other hand, have been found to complete more household-serving travel
than men, such as escorting children
\citep{craigGenderMobilityParental2019, taylorWhatExplainsGender2015, hanTaskAllocationGender2019, mcdonaldExploratoryAnalysisChildren2006},
shopping, and errand trips
\citep{taylorWhatExplainsGender2015, rootWomenTravelIdea2000, sweetGenderDifferencesRole2016}.

Although research on the gendered distribution of household-serving
travel has existed for decades, it was Sánchez de Madariaga who
introduced the ``Mobility of Care'' framework to support the proper
accounting of travel needed to fulfill caring and home-related
activities (e.g., the combined travel to grocery stores, errands, and
picking-up or dropping off children)
\citep{sanchezdemadariagaMobilityCareIntroducing2013}. Mobility of Care
highlights how household-serving travel is systematically
under-represented, under-counted, and rendered invisible in transport
planning, particularly in travel surveys. Travel surveys are a key
source of mobility data for transportation planners in metropolitan
cities, and their primary focus is on the collection of `compulsory'
trip purposes such as school and work. In the Canadian context,
respondents of the Transportation Tomorrow Survey (TTS) which
encompasses the cities of Toronto, Hamilton and surrounding urban area
\citep{transportationtomorrowsurvey2018}, are given the following
options to categorize their trip origins and destinations: home, work,
school, daycare, facilitate passenger, marketing/shopping, other, or
unknown. While home-work and home-school trips are easily identified,
care trips are more challenging to discern. Likely, many shopping trips
are for care purposes (e.g., groceries), but others may be for leisure.
While escort trips may be well captured under the categories `daycare'
or `facilitate passenger', trips to run errands or to attend health
appointments may not be; it is probable that respondents categorize many
of these trips as `other' or even `unknown'. Ultimately, the travel
survey's focus is on a `typical' trip to work or school
\citep{transportationtomorrowsurvey2018}; other trips are a by-product,
minimialised in importance. Of course, people's travel behaviours are
complex and surveys must balance detail with summary. However, what is
seen as a `typical' trip continues to shape transport and land-use, and
this aggregation steers data-driven solutions from counted and observed
home -work/-school based trips.

When travel surveys \emph{are} designed to explicitly capture mobility
of care, preliminary research has found that it comprises approximately
one third of adults' trips
\citep{gomezvaroAccountingCareEveryday2023, sanchezdemadariagaMobilityCareIntroducing2013, sanchezdemadariagaMeasuringMobilitiesCare2019, ravensbergenExploratoryAnalysisMobility2022, murillomunarCaregiversMoveGender2023}.
Given the large proportion of daily travel that mobility of care
comprises, these trips should be explicitly captured in transport
research. Further, the current under-reporting of mobility of care in
research and planning has important equity considerations. Not only are
mobility of care trips completed predominantly by women, this gendered
discrepancy is greater in low-income households
\citep{murillomunarCaregiversMoveGender2023, sanchezdemadariagaMobilityCareIntroducing2013, ravensbergenExploratoryAnalysisMobility2022}.
For instance, in lower income households in the city of Montréal, women
complete 50\% more care trips than men
\citep{ravensbergenExploratoryAnalysisMobility2022}. The power of the
Mobility of Care concept lies in its ability to highlight the
masculinist bias in transport research -- travel for care appears
insignificant because travel surveys are not written to capture it
\citep{sanchezdemadariagaMobilityCareIntroducing2013}.

Travel surveys, however, are but one source of information used by
transport researchers and practitioners. Another popular instrument is
accessibility, especially in the case of sustainable and equitable
cities
\citep{ryanAccessibilitySpaceTime2023, bertoliniSustainableAccessibilityConceptual2005}.
Accessibility is an indicator of the ease of interacting with
destinations. However, the point of interest in many accessibility-based
assessments has been travel to work destinations by car or public
transit modes e.g.,
\citep{kelobonyeRelativeAccessibilityAnalysis2019, farberOntarioLineSocioeconomic2019, duarteInfluenceJobAccessibility2023, ryanAccessibilitySpaceTime2023}.
Jobs are not always the most significant destination for many segments
of the population. Further, modal options to employment and care trips
differ. For example, women's commutes are on average a smaller
proportion of their daily travel than men's
\citep{ravensbergenExploratoryAnalysisMobility2022}. Care trips are also
less likely to be completed by public transit or bicycle and are more
likely done by car or by foot than the commute
\citep{ravensbergenExploratoryAnalysisMobility2022}. One way to apply a
gender-aware lens to accessibility analysis is by explicitly considering
access to destinations involved in Mobility of Care by multiple modes.
Reframing accessibility analysis in this way reinforces its importance
as an instrument that supports the planning of sustainable and equitable
travel and land-use in cities.

Taken together, this study's objective is to contribute to the transport
planning literature through the demonstration of a multimodal
accessibility analysis of Mobility of Care destinations. Two
accessibility measures are used: the cumulative opportunities measure
and the spatial availability measure. The measures are applied on a care
destination dataset with novel Mobility of Care classifications for the
city of Hamilton, Canada. The potential access to Mobility of Care
destinations for walking, transit, bike, and on foot is calculated for
15- and 30-minute travel time thresholds. Results are compared across
the two measures and four modes, and the overlap between low
accessibility areas and high low-income prevalence is presented.
Implications of the results are discussed along with study conclusions.

\section{Overview of multimodal accessibility
analysis}\label{overview-of-multimodal-accessibility-analysis}

As indicators of ``the potential of opportunities for interaction''
\citep{hansenHowAccessibilityShapes1959}, accessibility measures can
also be interpreted as the relative ease of reaching destinations using
transport networks: they are a byproduct of mobility and a
representation of people's interaction with land-use and transportation
systems
\citep{hansenHowAccessibilityShapes1959, handyAccessibilityIdeaWhose2020, elgeneidyMakingAccessibilityWork2021}.

The cumulative opportunities measure is a popular accessibility measure,
widely appreciated for its intuitive computation
\citep{handyAccessibilityIdeaWhose2020, handyMeasuringAccessibilityExploration1997, kelobonyeRelativeAccessibilityAnalysis2019, chengInvestigatingWalkingAccessibility2019}.
It quantifies how many destinations can be reached from a point in space
within a given travel time threshold. The measure has been used to
quantify access, given a travel time threshold and mode, often to
employment destinations. Namely, access to employment is explored by car
and/or transit
\citep{kapatsilaResolvingAccessibilityDilemma2023, deboosereEvaluatingEquityAccessibility2018, tomasiello2023time},
by bike \citep{imani2019cycle}, and by foot \citep{singh2022cumulative}.
Non-work amenities have also been analysed by this popular measure. For
example, grocery stores \citep{hosford15minuteCityReach2022} and
`baskets' of urban-amenities
\citep{mccahill2018non, klumpenhouwer2021flexible, chengInvestigatingWalkingAccessibility2019}.
From the authors' review, the cumulative opportunities literature has
not yet focused on destination selection from the lens of Mobility of
Care.

A critique leveled at the cumulative opportunities measure (and other
non-competitive accessibility measures) is its omission of
competition-for-opportunities effects
\citep{paezDemand2019, soukhovIntroducingSpatialAvailability2023, kelobonyeMeasuringAccessibilitySpatial2020, merlinDoesCompetitionMatter2017}.
Conceptually, this consideration is important as opportunities are
finite, which leads to competition between the population seeking them.
However, planners often opt for simpler measures
\citep{kapatsilaResolvingAccessibilityDilemma2023} as measures that
account for competition tend to be more difficult to implement and
interpret \citep{merlinDoesCompetitionMatter2017}. In the recent work of
\citet{soukhovIntroducingSpatialAvailability2023}, an accessibility
measure named Spatial Availability is introduced that simplifies the
interpretation of resulting values while considering competition using a
proportional allocation. It is extended for multimodal applications in
\citet{soukhovMultimodalSpatialAvailability2024}. Notably, the use of
competitive accessibility measures to explore access to a variety of
destinations is scarce, with only recent exceptions (e.g.,
\citet{kelobonyeMeasuringAccessibilitySpatial2020} and
\citet{singh2022cumulative}). Moreover, competitive accessibility
measures have yet to be focused on Mobility of Care destinations.

As presented in this work, two multimodal accessibility measures are
implemented for the calculation of accessibility to Mobility of Care
destinations. The first is a routinely used measure, the cumulative
opportunities measure, and the second is a competitive and
singly-constrained measure, spatial availability
\citep{soukhovIntroducingSpatialAvailability2023, soukhovMultimodalSpatialAvailability2024}.

\section{Background on Hamilton}\label{background-on-hamilton}

This paper focuses on Hamilton as a case study, a mid-size city of
approximately 500,000 residents that lies within the urban and suburban
Greater Toronto and Hamilton Area and is home to seven million people,
or approximately 20\% of the Canadian population
\citep{cityoftoronto2021CensusPopulationa2022}.

Hamilton is divided into six regional communities
(Figure~\ref{fig-Fig1}). Hamilton-Central is the most urbanized of the
six, and the five periphery communities of Dundas, Ancaster,
Flamborough, Glanbrook and Stoney Creek are significantly more
suburbanized with the furthest periphery regions being undeveloped or
rural owing to their inclusion in the region's greenbelt
\citep{greenbeltfoundationThrivingGreenbeltThriving2023}. These
different urban forms and associated transport infrastructure play a key
role in access to care destinations. Hamilton Street Railway (HSR) is
the city's transit provider operating only buses at the current date.
Notably, Hamilton-Central is the only community fully serviced by HSR
and has the highest concentration of walking and bike infrastructure for
mainstream use (e.g., Level of Traffic Stress 1 or 2 which indicates
low-speed, low-volume streets, separated bicycle facilities, and
dedicated lanes where cyclist must interact with traffic at formal
crossings \citep{conveyalCyclingLevelTraffic2024}) as identified in the
OpenStreetMaps road network \citep{geofabrikOntarioCanadaOpen2023} and
the city's General Transit Feed Specification file
\citep{transitfeedsHamiltonStreetRailway2023}.

\begin{figure}

\centering{

\includegraphics[width=6.25in,height=\textheight]{figures/Fig1-descriptive-boundaries.png}

}

\caption{\label{fig-Fig1}The six former muncipal boundaries in the city
of Hamilton (green), highways and arerial roads (grey), walking and
cycling infrastructure (light grey), and concentration of transit bus
stops (reds). Geographic layer sources: road network
\citep{geofabrikOntarioCanadaOpen2023}, transit stops
\citep{transitfeedsHamiltonStreetRailway2023}, community boundaries
\citep{opendatahamiltonCityBoundary2023} and lake
\citep{greatlakesUSGS2010}.}

\end{figure}%

\subsection{Care destination dataset}\label{care-destination-dataset}

A novel geospatial dataset of care destinations for Hamilton was
compiled using a variety of local sources and manually confirmed through
Google Maps. As a way to showcase the dataset, it is grouped by care
destination category. These five categories were generated by the
authors following the travel purpose categories created in the mobility
of care research by
\citet{sanchezdemadariagaMeasuringMobilitiesCare2019}. Notably:
child-centric (destinations for ``Childcare'' escorting trips),
elder-centric (common destinations for other escorting trips that are
not childcare-focused), grocery-centric, health-centric, and
errand-centric destinations. The majority of destinations included can
be publicly accessed (e.g., only public schools, grocery stores,
clinics, community centres). However, certain destinations may require a
fee that could be prohibitive for lower-income households (e.g., all
long term care homes, both publicly subsided or private are included).
Category sources of data and preparation notes are detailed in
Table~\ref{tbl-Tbl1}. Their spatial distribution and sub-categories are
visualised in Figure~\ref{fig-Fig2}.

\global\setlength{\Oldarrayrulewidth}{\arrayrulewidth}

\global\setlength{\Oldtabcolsep}{\tabcolsep}

\setlength{\tabcolsep}{2pt}

\renewcommand*{\arraystretch}{1.5}



\providecommand{\ascline}[3]{\noalign{\global\arrayrulewidth #1}\arrayrulecolor[HTML]{#2}\cline{#3}}

\begin{longtable}[c]{|p{0.67in}|p{1.69in}|p{4.33in}}

\caption{\label{tbl-Tbl1}Details on the preparation and data sources of
care destinations.}

\tabularnewline

\ascline{1.5pt}{666666}{1-3}

\multicolumn{1}{>{\raggedright}m{\dimexpr 0.67in+0\tabcolsep}}{\textcolor[HTML]{000000}{\fontsize{10}{10}\selectfont{\global\setmainfont{Arial}{Care\ category}}}} & \multicolumn{1}{>{\raggedright}m{\dimexpr 1.69in+0\tabcolsep}}{\textcolor[HTML]{000000}{\fontsize{10}{10}\selectfont{\global\setmainfont{Arial}{Sources}}}} & \multicolumn{1}{>{\raggedright}m{\dimexpr 4.33in+0\tabcolsep}}{\textcolor[HTML]{000000}{\fontsize{10}{10}\selectfont{\global\setmainfont{Arial}{Data\ preparation\ notes}}}} \\

\ascline{1.5pt}{666666}{1-3}\endfirsthead 

\ascline{1.5pt}{666666}{1-3}

\multicolumn{1}{>{\raggedright}m{\dimexpr 0.67in+0\tabcolsep}}{\textcolor[HTML]{000000}{\fontsize{10}{10}\selectfont{\global\setmainfont{Arial}{Care\ category}}}} & \multicolumn{1}{>{\raggedright}m{\dimexpr 1.69in+0\tabcolsep}}{\textcolor[HTML]{000000}{\fontsize{10}{10}\selectfont{\global\setmainfont{Arial}{Sources}}}} & \multicolumn{1}{>{\raggedright}m{\dimexpr 4.33in+0\tabcolsep}}{\textcolor[HTML]{000000}{\fontsize{10}{10}\selectfont{\global\setmainfont{Arial}{Data\ preparation\ notes}}}} \\

\ascline{1.5pt}{666666}{1-3}\endhead



\multicolumn{1}{>{\raggedright}m{\dimexpr 0.67in+0\tabcolsep}}{\textcolor[HTML]{000000}{\fontsize{10}{10}\selectfont{\global\setmainfont{Arial}{Child-centric}}}} & \multicolumn{1}{>{\raggedright}m{\dimexpr 1.69in+0\tabcolsep}}{\textcolor[HTML]{000000}{\fontsize{10}{10}\selectfont{\global\setmainfont{Arial}{(Hamilton}}}\textcolor[HTML]{000000}{\fontsize{10}{10}\selectfont{\global\setmainfont{Arial}{\ }}}\textcolor[HTML]{000000}{\fontsize{10}{10}\selectfont{\global\setmainfont{Arial}{2022a,}}}\textcolor[HTML]{000000}{\fontsize{10}{10}\selectfont{\global\setmainfont{Arial}{\ }}}\textcolor[HTML]{000000}{\fontsize{10}{10}\selectfont{\global\setmainfont{Arial}{2023,}}}\textcolor[HTML]{000000}{\fontsize{10}{10}\selectfont{\global\setmainfont{Arial}{\ }}}\textcolor[HTML]{000000}{\fontsize{10}{10}\selectfont{\global\setmainfont{Arial}{2022c,}}}\textcolor[HTML]{000000}{\fontsize{10}{10}\selectfont{\global\setmainfont{Arial}{\ }}}\textcolor[HTML]{000000}{\fontsize{10}{10}\selectfont{\global\setmainfont{Arial}{2022d;}}}\textcolor[HTML]{000000}{\fontsize{10}{10}\selectfont{\global\setmainfont{Arial}{\ }}}\textcolor[HTML]{000000}{\fontsize{10}{10}\selectfont{\global\setmainfont{Arial}{Ontario}}}\textcolor[HTML]{000000}{\fontsize{10}{10}\selectfont{\global\setmainfont{Arial}{\ }}}\textcolor[HTML]{000000}{\fontsize{10}{10}\selectfont{\global\setmainfont{Arial}{2023b)}}}} & \multicolumn{1}{>{\raggedright}m{\dimexpr 4.33in+0\tabcolsep}}{\textcolor[HTML]{000000}{\fontsize{10}{10}\selectfont{\global\setmainfont{Arial}{Public}}}\textcolor[HTML]{000000}{\fontsize{10}{10}\selectfont{\global\setmainfont{Arial}{\ }}}\textcolor[HTML]{000000}{\fontsize{10}{10}\selectfont{\global\setmainfont{Arial}{schools,}}}\textcolor[HTML]{000000}{\fontsize{10}{10}\selectfont{\global\setmainfont{Arial}{\ }}}\textcolor[HTML]{000000}{\fontsize{10}{10}\selectfont{\global\setmainfont{Arial}{public}}}\textcolor[HTML]{000000}{\fontsize{10}{10}\selectfont{\global\setmainfont{Arial}{\ }}}\textcolor[HTML]{000000}{\fontsize{10}{10}\selectfont{\global\setmainfont{Arial}{and}}}\textcolor[HTML]{000000}{\fontsize{10}{10}\selectfont{\global\setmainfont{Arial}{\ }}}\textcolor[HTML]{000000}{\fontsize{10}{10}\selectfont{\global\setmainfont{Arial}{private}}}\textcolor[HTML]{000000}{\fontsize{10}{10}\selectfont{\global\setmainfont{Arial}{\ }}}\textcolor[HTML]{000000}{\fontsize{10}{10}\selectfont{\global\setmainfont{Arial}{(licensed)}}}\textcolor[HTML]{000000}{\fontsize{10}{10}\selectfont{\global\setmainfont{Arial}{\ }}}\textcolor[HTML]{000000}{\fontsize{10}{10}\selectfont{\global\setmainfont{Arial}{daycares,}}}\textcolor[HTML]{000000}{\fontsize{10}{10}\selectfont{\global\setmainfont{Arial}{\ }}}\textcolor[HTML]{000000}{\fontsize{10}{10}\selectfont{\global\setmainfont{Arial}{and}}}\textcolor[HTML]{000000}{\fontsize{10}{10}\selectfont{\global\setmainfont{Arial}{\ }}}\textcolor[HTML]{000000}{\fontsize{10}{10}\selectfont{\global\setmainfont{Arial}{public}}}\textcolor[HTML]{000000}{\fontsize{10}{10}\selectfont{\global\setmainfont{Arial}{\ }}}\textcolor[HTML]{000000}{\fontsize{10}{10}\selectfont{\global\setmainfont{Arial}{community}}}\textcolor[HTML]{000000}{\fontsize{10}{10}\selectfont{\global\setmainfont{Arial}{\ }}}\textcolor[HTML]{000000}{\fontsize{10}{10}\selectfont{\global\setmainfont{Arial}{centres,}}}\textcolor[HTML]{000000}{\fontsize{10}{10}\selectfont{\global\setmainfont{Arial}{\ }}}\textcolor[HTML]{000000}{\fontsize{10}{10}\selectfont{\global\setmainfont{Arial}{public}}}\textcolor[HTML]{000000}{\fontsize{10}{10}\selectfont{\global\setmainfont{Arial}{\ }}}\textcolor[HTML]{000000}{\fontsize{10}{10}\selectfont{\global\setmainfont{Arial}{recreation}}}\textcolor[HTML]{000000}{\fontsize{10}{10}\selectfont{\global\setmainfont{Arial}{\ }}}\textcolor[HTML]{000000}{\fontsize{10}{10}\selectfont{\global\setmainfont{Arial}{centres,}}}\textcolor[HTML]{000000}{\fontsize{10}{10}\selectfont{\global\setmainfont{Arial}{\ }}}\textcolor[HTML]{000000}{\fontsize{10}{10}\selectfont{\global\setmainfont{Arial}{and}}}\textcolor[HTML]{000000}{\fontsize{10}{10}\selectfont{\global\setmainfont{Arial}{\ }}}\textcolor[HTML]{000000}{\fontsize{10}{10}\selectfont{\global\setmainfont{Arial}{public}}}\textcolor[HTML]{000000}{\fontsize{10}{10}\selectfont{\global\setmainfont{Arial}{\ }}}\textcolor[HTML]{000000}{\fontsize{10}{10}\selectfont{\global\setmainfont{Arial}{parks:}}}\textcolor[HTML]{000000}{\fontsize{10}{10}\selectfont{\global\setmainfont{Arial}{\ }}}\textcolor[HTML]{000000}{\fontsize{10}{10}\selectfont{\global\setmainfont{Arial}{1,190}}}\textcolor[HTML]{000000}{\fontsize{10}{10}\selectfont{\global\setmainfont{Arial}{\ }}}\textcolor[HTML]{000000}{\fontsize{10}{10}\selectfont{\global\setmainfont{Arial}{locations}}}\textcolor[HTML]{000000}{\fontsize{10}{10}\selectfont{\global\setmainfont{Arial}{\ }}}\textcolor[HTML]{000000}{\fontsize{10}{10}\selectfont{\global\setmainfont{Arial}{are}}}\textcolor[HTML]{000000}{\fontsize{10}{10}\selectfont{\global\setmainfont{Arial}{\ }}}\textcolor[HTML]{000000}{\fontsize{10}{10}\selectfont{\global\setmainfont{Arial}{included.}}}\textcolor[HTML]{000000}{\fontsize{10}{10}\selectfont{\global\setmainfont{Arial}{\ }}}\textcolor[HTML]{000000}{\fontsize{10}{10}\selectfont{\global\setmainfont{Arial}{After}}}\textcolor[HTML]{000000}{\fontsize{10}{10}\selectfont{\global\setmainfont{Arial}{\ }}}\textcolor[HTML]{000000}{\fontsize{10}{10}\selectfont{\global\setmainfont{Arial}{manual}}}\textcolor[HTML]{000000}{\fontsize{10}{10}\selectfont{\global\setmainfont{Arial}{\ }}}\textcolor[HTML]{000000}{\fontsize{10}{10}\selectfont{\global\setmainfont{Arial}{review,}}}\textcolor[HTML]{000000}{\fontsize{10}{10}\selectfont{\global\setmainfont{Arial}{\ }}}\textcolor[HTML]{000000}{\fontsize{10}{10}\selectfont{\global\setmainfont{Arial}{all}}}\textcolor[HTML]{000000}{\fontsize{10}{10}\selectfont{\global\setmainfont{Arial}{\ }}}\textcolor[HTML]{000000}{\fontsize{10}{10}\selectfont{\global\setmainfont{Arial}{locations}}}\textcolor[HTML]{000000}{\fontsize{10}{10}\selectfont{\global\setmainfont{Arial}{\ }}}\textcolor[HTML]{000000}{\fontsize{10}{10}\selectfont{\global\setmainfont{Arial}{that}}}\textcolor[HTML]{000000}{\fontsize{10}{10}\selectfont{\global\setmainfont{Arial}{\ }}}\textcolor[HTML]{000000}{\fontsize{10}{10}\selectfont{\global\setmainfont{Arial}{typically}}}\textcolor[HTML]{000000}{\fontsize{10}{10}\selectfont{\global\setmainfont{Arial}{\ }}}\textcolor[HTML]{000000}{\fontsize{10}{10}\selectfont{\global\setmainfont{Arial}{do}}}\textcolor[HTML]{000000}{\fontsize{10}{10}\selectfont{\global\setmainfont{Arial}{\ }}}\textcolor[HTML]{000000}{\fontsize{10}{10}\selectfont{\global\setmainfont{Arial}{not}}}\textcolor[HTML]{000000}{\fontsize{10}{10}\selectfont{\global\setmainfont{Arial}{\ }}}\textcolor[HTML]{000000}{\fontsize{10}{10}\selectfont{\global\setmainfont{Arial}{serve}}}\textcolor[HTML]{000000}{\fontsize{10}{10}\selectfont{\global\setmainfont{Arial}{\ }}}\textcolor[HTML]{000000}{\fontsize{10}{10}\selectfont{\global\setmainfont{Arial}{children}}}\textcolor[HTML]{000000}{\fontsize{10}{10}\selectfont{\global\setmainfont{Arial}{\ }}}\textcolor[HTML]{000000}{\fontsize{10}{10}\selectfont{\global\setmainfont{Arial}{were}}}\textcolor[HTML]{000000}{\fontsize{10}{10}\selectfont{\global\setmainfont{Arial}{\ }}}\textcolor[HTML]{000000}{\fontsize{10}{10}\selectfont{\global\setmainfont{Arial}{removed}}}\textcolor[HTML]{000000}{\fontsize{10}{10}\selectfont{\global\setmainfont{Arial}{\ }}}\textcolor[HTML]{000000}{\fontsize{10}{10}\selectfont{\global\setmainfont{Arial}{including:}}}\textcolor[HTML]{000000}{\fontsize{10}{10}\selectfont{\global\setmainfont{Arial}{\ }}}\textcolor[HTML]{000000}{\fontsize{10}{10}\selectfont{\global\setmainfont{Arial}{Post-Secondary,}}}\textcolor[HTML]{000000}{\fontsize{10}{10}\selectfont{\global\setmainfont{Arial}{\ }}}\textcolor[HTML]{000000}{\fontsize{10}{10}\selectfont{\global\setmainfont{Arial}{Adult-Learning}}}\textcolor[HTML]{000000}{\fontsize{10}{10}\selectfont{\global\setmainfont{Arial}{\ }}}\textcolor[HTML]{000000}{\fontsize{10}{10}\selectfont{\global\setmainfont{Arial}{Centres,}}}\textcolor[HTML]{000000}{\fontsize{10}{10}\selectfont{\global\setmainfont{Arial}{\ }}}\textcolor[HTML]{000000}{\fontsize{10}{10}\selectfont{\global\setmainfont{Arial}{Group}}}\textcolor[HTML]{000000}{\fontsize{10}{10}\selectfont{\global\setmainfont{Arial}{\ }}}\textcolor[HTML]{000000}{\fontsize{10}{10}\selectfont{\global\setmainfont{Arial}{Homes,}}}\textcolor[HTML]{000000}{\fontsize{10}{10}\selectfont{\global\setmainfont{Arial}{\ }}}\textcolor[HTML]{000000}{\fontsize{10}{10}\selectfont{\global\setmainfont{Arial}{and}}}\textcolor[HTML]{000000}{\fontsize{10}{10}\selectfont{\global\setmainfont{Arial}{\ }}}\textcolor[HTML]{000000}{\fontsize{10}{10}\selectfont{\global\setmainfont{Arial}{Foster}}}\textcolor[HTML]{000000}{\fontsize{10}{10}\selectfont{\global\setmainfont{Arial}{\ }}}\textcolor[HTML]{000000}{\fontsize{10}{10}\selectfont{\global\setmainfont{Arial}{Care}}}\textcolor[HTML]{000000}{\fontsize{10}{10}\selectfont{\global\setmainfont{Arial}{\ }}}\textcolor[HTML]{000000}{\fontsize{10}{10}\selectfont{\global\setmainfont{Arial}{Centres.}}}\textcolor[HTML]{000000}{\fontsize{10}{10}\selectfont{\global\setmainfont{Arial}{\ }}}\textcolor[HTML]{000000}{\fontsize{10}{10}\selectfont{\global\setmainfont{Arial}{Further,}}}\textcolor[HTML]{000000}{\fontsize{10}{10}\selectfont{\global\setmainfont{Arial}{\ }}}\textcolor[HTML]{000000}{\fontsize{10}{10}\selectfont{\global\setmainfont{Arial}{through}}}\textcolor[HTML]{000000}{\fontsize{10}{10}\selectfont{\global\setmainfont{Arial}{\ }}}\textcolor[HTML]{000000}{\fontsize{10}{10}\selectfont{\global\setmainfont{Arial}{examination}}}\textcolor[HTML]{000000}{\fontsize{10}{10}\selectfont{\global\setmainfont{Arial}{\ }}}\textcolor[HTML]{000000}{\fontsize{10}{10}\selectfont{\global\setmainfont{Arial}{some}}}\textcolor[HTML]{000000}{\fontsize{10}{10}\selectfont{\global\setmainfont{Arial}{\ }}}\textcolor[HTML]{000000}{\fontsize{10}{10}\selectfont{\global\setmainfont{Arial}{Section}}}\textcolor[HTML]{000000}{\fontsize{10}{10}\selectfont{\global\setmainfont{Arial}{\ }}}\textcolor[HTML]{000000}{\fontsize{10}{10}\selectfont{\global\setmainfont{Arial}{23}}}\textcolor[HTML]{000000}{\fontsize{10}{10}\selectfont{\global\setmainfont{Arial}{\ }}}\textcolor[HTML]{000000}{\fontsize{10}{10}\selectfont{\global\setmainfont{Arial}{institutions}}}\textcolor[HTML]{000000}{\fontsize{10}{10}\selectfont{\global\setmainfont{Arial}{\ }}}\textcolor[HTML]{000000}{\fontsize{10}{10}\selectfont{\global\setmainfont{Arial}{defined}}}\textcolor[HTML]{000000}{\fontsize{10}{10}\selectfont{\global\setmainfont{Arial}{\ }}}\textcolor[HTML]{000000}{\fontsize{10}{10}\selectfont{\global\setmainfont{Arial}{as}}}\textcolor[HTML]{000000}{\fontsize{10}{10}\selectfont{\global\setmainfont{Arial}{\ }}}\textcolor[HTML]{000000}{\fontsize{10}{10}\selectfont{\global\setmainfont{Arial}{\textit{“}}}}\textcolor[HTML]{000000}{\fontsize{10}{10}\selectfont{\global\setmainfont{Arial}{\textit{centres}}}}\textcolor[HTML]{000000}{\fontsize{10}{10}\selectfont{\global\setmainfont{Arial}{\textit{\ }}}}\textcolor[HTML]{000000}{\fontsize{10}{10}\selectfont{\global\setmainfont{Arial}{\textit{for}}}}\textcolor[HTML]{000000}{\fontsize{10}{10}\selectfont{\global\setmainfont{Arial}{\textit{\ }}}}\textcolor[HTML]{000000}{\fontsize{10}{10}\selectfont{\global\setmainfont{Arial}{\textit{children}}}}\textcolor[HTML]{000000}{\fontsize{10}{10}\selectfont{\global\setmainfont{Arial}{\textit{\ }}}}\textcolor[HTML]{000000}{\fontsize{10}{10}\selectfont{\global\setmainfont{Arial}{\textit{who}}}}\textcolor[HTML]{000000}{\fontsize{10}{10}\selectfont{\global\setmainfont{Arial}{\textit{\ }}}}\textcolor[HTML]{000000}{\fontsize{10}{10}\selectfont{\global\setmainfont{Arial}{\textit{cannot}}}}\textcolor[HTML]{000000}{\fontsize{10}{10}\selectfont{\global\setmainfont{Arial}{\textit{\ }}}}\textcolor[HTML]{000000}{\fontsize{10}{10}\selectfont{\global\setmainfont{Arial}{\textit{attend}}}}\textcolor[HTML]{000000}{\fontsize{10}{10}\selectfont{\global\setmainfont{Arial}{\textit{\ }}}}\textcolor[HTML]{000000}{\fontsize{10}{10}\selectfont{\global\setmainfont{Arial}{\textit{school}}}}\textcolor[HTML]{000000}{\fontsize{10}{10}\selectfont{\global\setmainfont{Arial}{\textit{\ }}}}\textcolor[HTML]{000000}{\fontsize{10}{10}\selectfont{\global\setmainfont{Arial}{\textit{to}}}}\textcolor[HTML]{000000}{\fontsize{10}{10}\selectfont{\global\setmainfont{Arial}{\textit{\ }}}}\textcolor[HTML]{000000}{\fontsize{10}{10}\selectfont{\global\setmainfont{Arial}{\textit{meet}}}}\textcolor[HTML]{000000}{\fontsize{10}{10}\selectfont{\global\setmainfont{Arial}{\textit{\ }}}}\textcolor[HTML]{000000}{\fontsize{10}{10}\selectfont{\global\setmainfont{Arial}{\textit{the}}}}\textcolor[HTML]{000000}{\fontsize{10}{10}\selectfont{\global\setmainfont{Arial}{\textit{\ }}}}\textcolor[HTML]{000000}{\fontsize{10}{10}\selectfont{\global\setmainfont{Arial}{\textit{needs}}}}\textcolor[HTML]{000000}{\fontsize{10}{10}\selectfont{\global\setmainfont{Arial}{\textit{\ }}}}\textcolor[HTML]{000000}{\fontsize{10}{10}\selectfont{\global\setmainfont{Arial}{\textit{of}}}}\textcolor[HTML]{000000}{\fontsize{10}{10}\selectfont{\global\setmainfont{Arial}{\textit{\ }}}}\textcolor[HTML]{000000}{\fontsize{10}{10}\selectfont{\global\setmainfont{Arial}{\textit{care}}}}\textcolor[HTML]{000000}{\fontsize{10}{10}\selectfont{\global\setmainfont{Arial}{\textit{\ }}}}\textcolor[HTML]{000000}{\fontsize{10}{10}\selectfont{\global\setmainfont{Arial}{\textit{or}}}}\textcolor[HTML]{000000}{\fontsize{10}{10}\selectfont{\global\setmainfont{Arial}{\textit{\ }}}}\textcolor[HTML]{000000}{\fontsize{10}{10}\selectfont{\global\setmainfont{Arial}{\textit{treatment,}}}}\textcolor[HTML]{000000}{\fontsize{10}{10}\selectfont{\global\setmainfont{Arial}{\textit{\ }}}}\textcolor[HTML]{000000}{\fontsize{10}{10}\selectfont{\global\setmainfont{Arial}{\textit{and}}}}\textcolor[HTML]{000000}{\fontsize{10}{10}\selectfont{\global\setmainfont{Arial}{\textit{\ }}}}\textcolor[HTML]{000000}{\fontsize{10}{10}\selectfont{\global\setmainfont{Arial}{\textit{rehabilitation}}}}\textcolor[HTML]{000000}{\fontsize{10}{10}\selectfont{\global\setmainfont{Arial}{\textit{”}}}}\textcolor[HTML]{000000}{\fontsize{10}{10}\selectfont{\global\setmainfont{Arial}{\ }}}\textcolor[HTML]{000000}{\fontsize{10}{10}\selectfont{\global\setmainfont{Arial}{(Ontario}}}\textcolor[HTML]{000000}{\fontsize{10}{10}\selectfont{\global\setmainfont{Arial}{\ }}}\textcolor[HTML]{000000}{\fontsize{10}{10}\selectfont{\global\setmainfont{Arial}{2023a)}}}\textcolor[HTML]{000000}{\fontsize{10}{10}\selectfont{\global\setmainfont{Arial}{,}}}\textcolor[HTML]{000000}{\fontsize{10}{10}\selectfont{\global\setmainfont{Arial}{\ }}}\textcolor[HTML]{000000}{\fontsize{10}{10}\selectfont{\global\setmainfont{Arial}{were}}}\textcolor[HTML]{000000}{\fontsize{10}{10}\selectfont{\global\setmainfont{Arial}{\ }}}\textcolor[HTML]{000000}{\fontsize{10}{10}\selectfont{\global\setmainfont{Arial}{kept}}}\textcolor[HTML]{000000}{\fontsize{10}{10}\selectfont{\global\setmainfont{Arial}{\ }}}\textcolor[HTML]{000000}{\fontsize{10}{10}\selectfont{\global\setmainfont{Arial}{due}}}\textcolor[HTML]{000000}{\fontsize{10}{10}\selectfont{\global\setmainfont{Arial}{\ }}}\textcolor[HTML]{000000}{\fontsize{10}{10}\selectfont{\global\setmainfont{Arial}{to}}}\textcolor[HTML]{000000}{\fontsize{10}{10}\selectfont{\global\setmainfont{Arial}{\ }}}\textcolor[HTML]{000000}{\fontsize{10}{10}\selectfont{\global\setmainfont{Arial}{their}}}\textcolor[HTML]{000000}{\fontsize{10}{10}\selectfont{\global\setmainfont{Arial}{\ }}}\textcolor[HTML]{000000}{\fontsize{10}{10}\selectfont{\global\setmainfont{Arial}{innate}}}\textcolor[HTML]{000000}{\fontsize{10}{10}\selectfont{\global\setmainfont{Arial}{\ }}}\textcolor[HTML]{000000}{\fontsize{10}{10}\selectfont{\global\setmainfont{Arial}{connection}}}\textcolor[HTML]{000000}{\fontsize{10}{10}\selectfont{\global\setmainfont{Arial}{\ }}}\textcolor[HTML]{000000}{\fontsize{10}{10}\selectfont{\global\setmainfont{Arial}{to}}}\textcolor[HTML]{000000}{\fontsize{10}{10}\selectfont{\global\setmainfont{Arial}{\ }}}\textcolor[HTML]{000000}{\fontsize{10}{10}\selectfont{\global\setmainfont{Arial}{care.}}}} \\





\multicolumn{1}{>{\raggedright}m{\dimexpr 0.67in+0\tabcolsep}}{\textcolor[HTML]{000000}{\fontsize{10}{10}\selectfont{\global\setmainfont{Arial}{Elder-centric}}}} & \multicolumn{1}{>{\raggedright}m{\dimexpr 1.69in+0\tabcolsep}}{\textcolor[HTML]{000000}{\fontsize{10}{10}\selectfont{\global\setmainfont{Arial}{(Hamilton}}}\textcolor[HTML]{000000}{\fontsize{10}{10}\selectfont{\global\setmainfont{Arial}{\ }}}\textcolor[HTML]{000000}{\fontsize{10}{10}\selectfont{\global\setmainfont{Arial}{2022d;}}}\textcolor[HTML]{000000}{\fontsize{10}{10}\selectfont{\global\setmainfont{Arial}{\ }}}\textcolor[HTML]{000000}{\fontsize{10}{10}\selectfont{\global\setmainfont{Arial}{Ontario GeoHub}}}\textcolor[HTML]{000000}{\fontsize{10}{10}\selectfont{\global\setmainfont{Arial}{\ }}}\textcolor[HTML]{000000}{\fontsize{10}{10}\selectfont{\global\setmainfont{Arial}{2023)}}}} & \multicolumn{1}{>{\raggedright}m{\dimexpr 4.33in+0\tabcolsep}}{\textcolor[HTML]{000000}{\fontsize{10}{10}\selectfont{\global\setmainfont{Arial}{Senior}}}\textcolor[HTML]{000000}{\fontsize{10}{10}\selectfont{\global\setmainfont{Arial}{\ }}}\textcolor[HTML]{000000}{\fontsize{10}{10}\selectfont{\global\setmainfont{Arial}{centres,}}}\textcolor[HTML]{000000}{\fontsize{10}{10}\selectfont{\global\setmainfont{Arial}{\ }}}\textcolor[HTML]{000000}{\fontsize{10}{10}\selectfont{\global\setmainfont{Arial}{long-term}}}\textcolor[HTML]{000000}{\fontsize{10}{10}\selectfont{\global\setmainfont{Arial}{\ }}}\textcolor[HTML]{000000}{\fontsize{10}{10}\selectfont{\global\setmainfont{Arial}{care}}}\textcolor[HTML]{000000}{\fontsize{10}{10}\selectfont{\global\setmainfont{Arial}{\ }}}\textcolor[HTML]{000000}{\fontsize{10}{10}\selectfont{\global\setmainfont{Arial}{homes,}}}\textcolor[HTML]{000000}{\fontsize{10}{10}\selectfont{\global\setmainfont{Arial}{\ }}}\textcolor[HTML]{000000}{\fontsize{10}{10}\selectfont{\global\setmainfont{Arial}{and}}}\textcolor[HTML]{000000}{\fontsize{10}{10}\selectfont{\global\setmainfont{Arial}{\ }}}\textcolor[HTML]{000000}{\fontsize{10}{10}\selectfont{\global\setmainfont{Arial}{retirement}}}\textcolor[HTML]{000000}{\fontsize{10}{10}\selectfont{\global\setmainfont{Arial}{\ }}}\textcolor[HTML]{000000}{\fontsize{10}{10}\selectfont{\global\setmainfont{Arial}{homes:}}}\textcolor[HTML]{000000}{\fontsize{10}{10}\selectfont{\global\setmainfont{Arial}{\ }}}\textcolor[HTML]{000000}{\fontsize{10}{10}\selectfont{\global\setmainfont{Arial}{75}}}\textcolor[HTML]{000000}{\fontsize{10}{10}\selectfont{\global\setmainfont{Arial}{\ }}}\textcolor[HTML]{000000}{\fontsize{10}{10}\selectfont{\global\setmainfont{Arial}{destinations}}}\textcolor[HTML]{000000}{\fontsize{10}{10}\selectfont{\global\setmainfont{Arial}{\ }}}\textcolor[HTML]{000000}{\fontsize{10}{10}\selectfont{\global\setmainfont{Arial}{are}}}\textcolor[HTML]{000000}{\fontsize{10}{10}\selectfont{\global\setmainfont{Arial}{\ }}}\textcolor[HTML]{000000}{\fontsize{10}{10}\selectfont{\global\setmainfont{Arial}{identified.}}}} \\





\multicolumn{1}{>{\raggedright}m{\dimexpr 0.67in+0\tabcolsep}}{\textcolor[HTML]{000000}{\fontsize{10}{10}\selectfont{\global\setmainfont{Arial}{Grocery-centric}}}} & \multicolumn{1}{>{\raggedright}m{\dimexpr 1.69in+0\tabcolsep}}{\textcolor[HTML]{000000}{\fontsize{10}{10}\selectfont{\global\setmainfont{Arial}{(Axle Data}}}\textcolor[HTML]{000000}{\fontsize{10}{10}\selectfont{\global\setmainfont{Arial}{\ }}}\textcolor[HTML]{000000}{\fontsize{10}{10}\selectfont{\global\setmainfont{Arial}{2023)}}}} & \multicolumn{1}{>{\raggedright}m{\dimexpr 4.33in+0\tabcolsep}}{\textcolor[HTML]{000000}{\fontsize{10}{10}\selectfont{\global\setmainfont{Arial}{Grocery}}}\textcolor[HTML]{000000}{\fontsize{10}{10}\selectfont{\global\setmainfont{Arial}{\ }}}\textcolor[HTML]{000000}{\fontsize{10}{10}\selectfont{\global\setmainfont{Arial}{stores,}}}\textcolor[HTML]{000000}{\fontsize{10}{10}\selectfont{\global\setmainfont{Arial}{\ }}}\textcolor[HTML]{000000}{\fontsize{10}{10}\selectfont{\global\setmainfont{Arial}{namely}}}\textcolor[HTML]{000000}{\fontsize{10}{10}\selectfont{\global\setmainfont{Arial}{\ }}}\textcolor[HTML]{000000}{\fontsize{10}{10}\selectfont{\global\setmainfont{Arial}{a}}}\textcolor[HTML]{000000}{\fontsize{10}{10}\selectfont{\global\setmainfont{Arial}{\ }}}\textcolor[HTML]{000000}{\fontsize{10}{10}\selectfont{\global\setmainfont{Arial}{place}}}\textcolor[HTML]{000000}{\fontsize{10}{10}\selectfont{\global\setmainfont{Arial}{\ }}}\textcolor[HTML]{000000}{\fontsize{10}{10}\selectfont{\global\setmainfont{Arial}{a}}}\textcolor[HTML]{000000}{\fontsize{10}{10}\selectfont{\global\setmainfont{Arial}{\ }}}\textcolor[HTML]{000000}{\fontsize{10}{10}\selectfont{\global\setmainfont{Arial}{household}}}\textcolor[HTML]{000000}{\fontsize{10}{10}\selectfont{\global\setmainfont{Arial}{\ }}}\textcolor[HTML]{000000}{\fontsize{10}{10}\selectfont{\global\setmainfont{Arial}{could}}}\textcolor[HTML]{000000}{\fontsize{10}{10}\selectfont{\global\setmainfont{Arial}{\ }}}\textcolor[HTML]{000000}{\fontsize{10}{10}\selectfont{\global\setmainfont{Arial}{buy}}}\textcolor[HTML]{000000}{\fontsize{10}{10}\selectfont{\global\setmainfont{Arial}{\ }}}\textcolor[HTML]{000000}{\fontsize{10}{10}\selectfont{\global\setmainfont{Arial}{groceries}}}\textcolor[HTML]{000000}{\fontsize{10}{10}\selectfont{\global\setmainfont{Arial}{\ }}}\textcolor[HTML]{000000}{\fontsize{10}{10}\selectfont{\global\setmainfont{Arial}{ranging}}}\textcolor[HTML]{000000}{\fontsize{10}{10}\selectfont{\global\setmainfont{Arial}{\ }}}\textcolor[HTML]{000000}{\fontsize{10}{10}\selectfont{\global\setmainfont{Arial}{from}}}\textcolor[HTML]{000000}{\fontsize{10}{10}\selectfont{\global\setmainfont{Arial}{\ }}}\textcolor[HTML]{000000}{\fontsize{10}{10}\selectfont{\global\setmainfont{Arial}{convenience}}}\textcolor[HTML]{000000}{\fontsize{10}{10}\selectfont{\global\setmainfont{Arial}{\ }}}\textcolor[HTML]{000000}{\fontsize{10}{10}\selectfont{\global\setmainfont{Arial}{stores}}}\textcolor[HTML]{000000}{\fontsize{10}{10}\selectfont{\global\setmainfont{Arial}{\ }}}\textcolor[HTML]{000000}{\fontsize{10}{10}\selectfont{\global\setmainfont{Arial}{to}}}\textcolor[HTML]{000000}{\fontsize{10}{10}\selectfont{\global\setmainfont{Arial}{\ }}}\textcolor[HTML]{000000}{\fontsize{10}{10}\selectfont{\global\setmainfont{Arial}{large}}}\textcolor[HTML]{000000}{\fontsize{10}{10}\selectfont{\global\setmainfont{Arial}{\ }}}\textcolor[HTML]{000000}{\fontsize{10}{10}\selectfont{\global\setmainfont{Arial}{retail}}}\textcolor[HTML]{000000}{\fontsize{10}{10}\selectfont{\global\setmainfont{Arial}{\ }}}\textcolor[HTML]{000000}{\fontsize{10}{10}\selectfont{\global\setmainfont{Arial}{stores:}}}\textcolor[HTML]{000000}{\fontsize{10}{10}\selectfont{\global\setmainfont{Arial}{\ }}}\textcolor[HTML]{000000}{\fontsize{10}{10}\selectfont{\global\setmainfont{Arial}{381}}}\textcolor[HTML]{000000}{\fontsize{10}{10}\selectfont{\global\setmainfont{Arial}{\ }}}\textcolor[HTML]{000000}{\fontsize{10}{10}\selectfont{\global\setmainfont{Arial}{destinations}}}\textcolor[HTML]{000000}{\fontsize{10}{10}\selectfont{\global\setmainfont{Arial}{\ }}}\textcolor[HTML]{000000}{\fontsize{10}{10}\selectfont{\global\setmainfont{Arial}{are}}}\textcolor[HTML]{000000}{\fontsize{10}{10}\selectfont{\global\setmainfont{Arial}{\ }}}\textcolor[HTML]{000000}{\fontsize{10}{10}\selectfont{\global\setmainfont{Arial}{identified.}}}\textcolor[HTML]{000000}{\fontsize{10}{10}\selectfont{\global\setmainfont{Arial}{\ }}}\textcolor[HTML]{000000}{\fontsize{10}{10}\selectfont{\global\setmainfont{Arial}{Data}}}\textcolor[HTML]{000000}{\fontsize{10}{10}\selectfont{\global\setmainfont{Arial}{\ }}}\textcolor[HTML]{000000}{\fontsize{10}{10}\selectfont{\global\setmainfont{Arial}{is}}}\textcolor[HTML]{000000}{\fontsize{10}{10}\selectfont{\global\setmainfont{Arial}{\ }}}\textcolor[HTML]{000000}{\fontsize{10}{10}\selectfont{\global\setmainfont{Arial}{filtered}}}\textcolor[HTML]{000000}{\fontsize{10}{10}\selectfont{\global\setmainfont{Arial}{\ }}}\textcolor[HTML]{000000}{\fontsize{10}{10}\selectfont{\global\setmainfont{Arial}{by}}}\textcolor[HTML]{000000}{\fontsize{10}{10}\selectfont{\global\setmainfont{Arial}{\ }}}\textcolor[HTML]{000000}{\fontsize{10}{10}\selectfont{\global\setmainfont{Arial}{Company}}}\textcolor[HTML]{000000}{\fontsize{10}{10}\selectfont{\global\setmainfont{Arial}{\ }}}\textcolor[HTML]{000000}{\fontsize{10}{10}\selectfont{\global\setmainfont{Arial}{Name,}}}\textcolor[HTML]{000000}{\fontsize{10}{10}\selectfont{\global\setmainfont{Arial}{\ }}}\textcolor[HTML]{000000}{\fontsize{10}{10}\selectfont{\global\setmainfont{Arial}{Suite}}}\textcolor[HTML]{000000}{\fontsize{10}{10}\selectfont{\global\setmainfont{Arial}{\ }}}\textcolor[HTML]{000000}{\fontsize{10}{10}\selectfont{\global\setmainfont{Arial}{Number,}}}\textcolor[HTML]{000000}{\fontsize{10}{10}\selectfont{\global\setmainfont{Arial}{\ }}}\textcolor[HTML]{000000}{\fontsize{10}{10}\selectfont{\global\setmainfont{Arial}{Address,}}}\textcolor[HTML]{000000}{\fontsize{10}{10}\selectfont{\global\setmainfont{Arial}{\ }}}\textcolor[HTML]{000000}{\fontsize{10}{10}\selectfont{\global\setmainfont{Arial}{City,}}}\textcolor[HTML]{000000}{\fontsize{10}{10}\selectfont{\global\setmainfont{Arial}{\ }}}\textcolor[HTML]{000000}{\fontsize{10}{10}\selectfont{\global\setmainfont{Arial}{Province,}}}\textcolor[HTML]{000000}{\fontsize{10}{10}\selectfont{\global\setmainfont{Arial}{\ }}}\textcolor[HTML]{000000}{\fontsize{10}{10}\selectfont{\global\setmainfont{Arial}{Phone}}}\textcolor[HTML]{000000}{\fontsize{10}{10}\selectfont{\global\setmainfont{Arial}{\ }}}\textcolor[HTML]{000000}{\fontsize{10}{10}\selectfont{\global\setmainfont{Arial}{Number}}}\textcolor[HTML]{000000}{\fontsize{10}{10}\selectfont{\global\setmainfont{Arial}{\ }}}\textcolor[HTML]{000000}{\fontsize{10}{10}\selectfont{\global\setmainfont{Arial}{and}}}\textcolor[HTML]{000000}{\fontsize{10}{10}\selectfont{\global\setmainfont{Arial}{\ }}}\textcolor[HTML]{000000}{\fontsize{10}{10}\selectfont{\global\setmainfont{Arial}{Postal}}}\textcolor[HTML]{000000}{\fontsize{10}{10}\selectfont{\global\setmainfont{Arial}{\ }}}\textcolor[HTML]{000000}{\fontsize{10}{10}\selectfont{\global\setmainfont{Arial}{Code.}}}\textcolor[HTML]{000000}{\fontsize{10}{10}\selectfont{\global\setmainfont{Arial}{\ }}}\textcolor[HTML]{000000}{\fontsize{10}{10}\selectfont{\global\setmainfont{Arial}{The}}}\textcolor[HTML]{000000}{\fontsize{10}{10}\selectfont{\global\setmainfont{Arial}{\ }}}\textcolor[HTML]{000000}{\fontsize{10}{10}\selectfont{\global\setmainfont{Arial}{type}}}\textcolor[HTML]{000000}{\fontsize{10}{10}\selectfont{\global\setmainfont{Arial}{\ }}}\textcolor[HTML]{000000}{\fontsize{10}{10}\selectfont{\global\setmainfont{Arial}{was}}}\textcolor[HTML]{000000}{\fontsize{10}{10}\selectfont{\global\setmainfont{Arial}{\ }}}\textcolor[HTML]{000000}{\fontsize{10}{10}\selectfont{\global\setmainfont{Arial}{then}}}\textcolor[HTML]{000000}{\fontsize{10}{10}\selectfont{\global\setmainfont{Arial}{\ }}}\textcolor[HTML]{000000}{\fontsize{10}{10}\selectfont{\global\setmainfont{Arial}{identified}}}\textcolor[HTML]{000000}{\fontsize{10}{10}\selectfont{\global\setmainfont{Arial}{\ }}}\textcolor[HTML]{000000}{\fontsize{10}{10}\selectfont{\global\setmainfont{Arial}{e.g.,}}}\textcolor[HTML]{000000}{\fontsize{10}{10}\selectfont{\global\setmainfont{Arial}{\ }}}\textcolor[HTML]{000000}{\fontsize{10}{10}\selectfont{\global\setmainfont{Arial}{grocers}}}\textcolor[HTML]{000000}{\fontsize{10}{10}\selectfont{\global\setmainfont{Arial}{\ }}}\textcolor[HTML]{000000}{\fontsize{10}{10}\selectfont{\global\setmainfont{Arial}{specialty}}}\textcolor[HTML]{000000}{\fontsize{10}{10}\selectfont{\global\setmainfont{Arial}{\ }}}\textcolor[HTML]{000000}{\fontsize{10}{10}\selectfont{\global\setmainfont{Arial}{foods,}}}\textcolor[HTML]{000000}{\fontsize{10}{10}\selectfont{\global\setmainfont{Arial}{\ }}}\textcolor[HTML]{000000}{\fontsize{10}{10}\selectfont{\global\setmainfont{Arial}{grocers}}}\textcolor[HTML]{000000}{\fontsize{10}{10}\selectfont{\global\setmainfont{Arial}{\ }}}\textcolor[HTML]{000000}{\fontsize{10}{10}\selectfont{\global\setmainfont{Arial}{retail,}}}\textcolor[HTML]{000000}{\fontsize{10}{10}\selectfont{\global\setmainfont{Arial}{\ }}}\textcolor[HTML]{000000}{\fontsize{10}{10}\selectfont{\global\setmainfont{Arial}{grocer}}}\textcolor[HTML]{000000}{\fontsize{10}{10}\selectfont{\global\setmainfont{Arial}{\ }}}\textcolor[HTML]{000000}{\fontsize{10}{10}\selectfont{\global\setmainfont{Arial}{health}}}\textcolor[HTML]{000000}{\fontsize{10}{10}\selectfont{\global\setmainfont{Arial}{\ }}}\textcolor[HTML]{000000}{\fontsize{10}{10}\selectfont{\global\setmainfont{Arial}{food,}}}\textcolor[HTML]{000000}{\fontsize{10}{10}\selectfont{\global\setmainfont{Arial}{\ }}}\textcolor[HTML]{000000}{\fontsize{10}{10}\selectfont{\global\setmainfont{Arial}{grocer}}}\textcolor[HTML]{000000}{\fontsize{10}{10}\selectfont{\global\setmainfont{Arial}{\ }}}\textcolor[HTML]{000000}{\fontsize{10}{10}\selectfont{\global\setmainfont{Arial}{wholesale,}}}\textcolor[HTML]{000000}{\fontsize{10}{10}\selectfont{\global\setmainfont{Arial}{\ }}}\textcolor[HTML]{000000}{\fontsize{10}{10}\selectfont{\global\setmainfont{Arial}{grocer}}}\textcolor[HTML]{000000}{\fontsize{10}{10}\selectfont{\global\setmainfont{Arial}{\ }}}\textcolor[HTML]{000000}{\fontsize{10}{10}\selectfont{\global\setmainfont{Arial}{curbside,}}}\textcolor[HTML]{000000}{\fontsize{10}{10}\selectfont{\global\setmainfont{Arial}{\ }}}\textcolor[HTML]{000000}{\fontsize{10}{10}\selectfont{\global\setmainfont{Arial}{grocer}}}\textcolor[HTML]{000000}{\fontsize{10}{10}\selectfont{\global\setmainfont{Arial}{\ }}}\textcolor[HTML]{000000}{\fontsize{10}{10}\selectfont{\global\setmainfont{Arial}{delicatessen}}}\textcolor[HTML]{000000}{\fontsize{10}{10}\selectfont{\global\setmainfont{Arial}{\ }}}\textcolor[HTML]{000000}{\fontsize{10}{10}\selectfont{\global\setmainfont{Arial}{wholesale,}}}\textcolor[HTML]{000000}{\fontsize{10}{10}\selectfont{\global\setmainfont{Arial}{\ }}}\textcolor[HTML]{000000}{\fontsize{10}{10}\selectfont{\global\setmainfont{Arial}{grocer}}}\textcolor[HTML]{000000}{\fontsize{10}{10}\selectfont{\global\setmainfont{Arial}{\ }}}\textcolor[HTML]{000000}{\fontsize{10}{10}\selectfont{\global\setmainfont{Arial}{convenience.}}}\textcolor[HTML]{000000}{\fontsize{10}{10}\selectfont{\global\setmainfont{Arial}{\ }}}\textcolor[HTML]{000000}{\fontsize{10}{10}\selectfont{\global\setmainfont{Arial}{Data}}}\textcolor[HTML]{000000}{\fontsize{10}{10}\selectfont{\global\setmainfont{Arial}{\ }}}\textcolor[HTML]{000000}{\fontsize{10}{10}\selectfont{\global\setmainfont{Arial}{was}}}\textcolor[HTML]{000000}{\fontsize{10}{10}\selectfont{\global\setmainfont{Arial}{\ }}}\textcolor[HTML]{000000}{\fontsize{10}{10}\selectfont{\global\setmainfont{Arial}{crossreferenced}}}\textcolor[HTML]{000000}{\fontsize{10}{10}\selectfont{\global\setmainfont{Arial}{\ }}}\textcolor[HTML]{000000}{\fontsize{10}{10}\selectfont{\global\setmainfont{Arial}{to}}}\textcolor[HTML]{000000}{\fontsize{10}{10}\selectfont{\global\setmainfont{Arial}{\ }}}\textcolor[HTML]{000000}{\fontsize{10}{10}\selectfont{\global\setmainfont{Arial}{ensure}}}\textcolor[HTML]{000000}{\fontsize{10}{10}\selectfont{\global\setmainfont{Arial}{\ }}}\textcolor[HTML]{000000}{\fontsize{10}{10}\selectfont{\global\setmainfont{Arial}{all}}}\textcolor[HTML]{000000}{\fontsize{10}{10}\selectfont{\global\setmainfont{Arial}{\ }}}\textcolor[HTML]{000000}{\fontsize{10}{10}\selectfont{\global\setmainfont{Arial}{included}}}\textcolor[HTML]{000000}{\fontsize{10}{10}\selectfont{\global\setmainfont{Arial}{\ }}}\textcolor[HTML]{000000}{\fontsize{10}{10}\selectfont{\global\setmainfont{Arial}{locations}}}\textcolor[HTML]{000000}{\fontsize{10}{10}\selectfont{\global\setmainfont{Arial}{\ }}}\textcolor[HTML]{000000}{\fontsize{10}{10}\selectfont{\global\setmainfont{Arial}{were}}}\textcolor[HTML]{000000}{\fontsize{10}{10}\selectfont{\global\setmainfont{Arial}{\ }}}\textcolor[HTML]{000000}{\fontsize{10}{10}\selectfont{\global\setmainfont{Arial}{operational}}}\textcolor[HTML]{000000}{\fontsize{10}{10}\selectfont{\global\setmainfont{Arial}{\ }}}\textcolor[HTML]{000000}{\fontsize{10}{10}\selectfont{\global\setmainfont{Arial}{and}}}\textcolor[HTML]{000000}{\fontsize{10}{10}\selectfont{\global\setmainfont{Arial}{\ }}}\textcolor[HTML]{000000}{\fontsize{10}{10}\selectfont{\global\setmainfont{Arial}{legitimate}}}\textcolor[HTML]{000000}{\fontsize{10}{10}\selectfont{\global\setmainfont{Arial}{\ }}}\textcolor[HTML]{000000}{\fontsize{10}{10}\selectfont{\global\setmainfont{Arial}{grocery}}}\textcolor[HTML]{000000}{\fontsize{10}{10}\selectfont{\global\setmainfont{Arial}{\ }}}\textcolor[HTML]{000000}{\fontsize{10}{10}\selectfont{\global\setmainfont{Arial}{stores.}}}} \\





\multicolumn{1}{>{\raggedright}m{\dimexpr 0.67in+0\tabcolsep}}{\textcolor[HTML]{000000}{\fontsize{10}{10}\selectfont{\global\setmainfont{Arial}{Health-centric}}}} & \multicolumn{1}{>{\raggedright}m{\dimexpr 1.69in+0\tabcolsep}}{\textcolor[HTML]{000000}{\fontsize{10}{10}\selectfont{\global\setmainfont{Arial}{(Ontario GeoHub}}}\textcolor[HTML]{000000}{\fontsize{10}{10}\selectfont{\global\setmainfont{Arial}{\ }}}\textcolor[HTML]{000000}{\fontsize{10}{10}\selectfont{\global\setmainfont{Arial}{2023;}}}\textcolor[HTML]{000000}{\fontsize{10}{10}\selectfont{\global\setmainfont{Arial}{\ }}}\textcolor[HTML]{000000}{\fontsize{10}{10}\selectfont{\global\setmainfont{Arial}{HNHB Healthline}}}\textcolor[HTML]{000000}{\fontsize{10}{10}\selectfont{\global\setmainfont{Arial}{\ }}}\textcolor[HTML]{000000}{\fontsize{10}{10}\selectfont{\global\setmainfont{Arial}{2023)}}}} & \multicolumn{1}{>{\raggedright}m{\dimexpr 4.33in+0\tabcolsep}}{\textcolor[HTML]{000000}{\fontsize{10}{10}\selectfont{\global\setmainfont{Arial}{Hospitals,}}}\textcolor[HTML]{000000}{\fontsize{10}{10}\selectfont{\global\setmainfont{Arial}{\ }}}\textcolor[HTML]{000000}{\fontsize{10}{10}\selectfont{\global\setmainfont{Arial}{pharmacies,}}}\textcolor[HTML]{000000}{\fontsize{10}{10}\selectfont{\global\setmainfont{Arial}{\ }}}\textcolor[HTML]{000000}{\fontsize{10}{10}\selectfont{\global\setmainfont{Arial}{clinics,}}}\textcolor[HTML]{000000}{\fontsize{10}{10}\selectfont{\global\setmainfont{Arial}{\ }}}\textcolor[HTML]{000000}{\fontsize{10}{10}\selectfont{\global\setmainfont{Arial}{and}}}\textcolor[HTML]{000000}{\fontsize{10}{10}\selectfont{\global\setmainfont{Arial}{\ }}}\textcolor[HTML]{000000}{\fontsize{10}{10}\selectfont{\global\setmainfont{Arial}{dentist}}}\textcolor[HTML]{000000}{\fontsize{10}{10}\selectfont{\global\setmainfont{Arial}{\ }}}\textcolor[HTML]{000000}{\fontsize{10}{10}\selectfont{\global\setmainfont{Arial}{offices:}}}\textcolor[HTML]{000000}{\fontsize{10}{10}\selectfont{\global\setmainfont{Arial}{\ }}}\textcolor[HTML]{000000}{\fontsize{10}{10}\selectfont{\global\setmainfont{Arial}{421}}}\textcolor[HTML]{000000}{\fontsize{10}{10}\selectfont{\global\setmainfont{Arial}{\ }}}\textcolor[HTML]{000000}{\fontsize{10}{10}\selectfont{\global\setmainfont{Arial}{destinations}}}\textcolor[HTML]{000000}{\fontsize{10}{10}\selectfont{\global\setmainfont{Arial}{\ }}}\textcolor[HTML]{000000}{\fontsize{10}{10}\selectfont{\global\setmainfont{Arial}{are}}}\textcolor[HTML]{000000}{\fontsize{10}{10}\selectfont{\global\setmainfont{Arial}{\ }}}\textcolor[HTML]{000000}{\fontsize{10}{10}\selectfont{\global\setmainfont{Arial}{identified.}}}\textcolor[HTML]{000000}{\fontsize{10}{10}\selectfont{\global\setmainfont{Arial}{\ }}}\textcolor[HTML]{000000}{\fontsize{10}{10}\selectfont{\global\setmainfont{Arial}{Hospitals}}}\textcolor[HTML]{000000}{\fontsize{10}{10}\selectfont{\global\setmainfont{Arial}{\ }}}\textcolor[HTML]{000000}{\fontsize{10}{10}\selectfont{\global\setmainfont{Arial}{and}}}\textcolor[HTML]{000000}{\fontsize{10}{10}\selectfont{\global\setmainfont{Arial}{\ }}}\textcolor[HTML]{000000}{\fontsize{10}{10}\selectfont{\global\setmainfont{Arial}{pharmacies}}}\textcolor[HTML]{000000}{\fontsize{10}{10}\selectfont{\global\setmainfont{Arial}{\ }}}\textcolor[HTML]{000000}{\fontsize{10}{10}\selectfont{\global\setmainfont{Arial}{were}}}\textcolor[HTML]{000000}{\fontsize{10}{10}\selectfont{\global\setmainfont{Arial}{\ }}}\textcolor[HTML]{000000}{\fontsize{10}{10}\selectfont{\global\setmainfont{Arial}{retrieved}}}\textcolor[HTML]{000000}{\fontsize{10}{10}\selectfont{\global\setmainfont{Arial}{\ }}}\textcolor[HTML]{000000}{\fontsize{10}{10}\selectfont{\global\setmainfont{Arial}{while}}}\textcolor[HTML]{000000}{\fontsize{10}{10}\selectfont{\global\setmainfont{Arial}{\ }}}\textcolor[HTML]{000000}{\fontsize{10}{10}\selectfont{\global\setmainfont{Arial}{clinics}}}\textcolor[HTML]{000000}{\fontsize{10}{10}\selectfont{\global\setmainfont{Arial}{\ }}}\textcolor[HTML]{000000}{\fontsize{10}{10}\selectfont{\global\setmainfont{Arial}{and}}}\textcolor[HTML]{000000}{\fontsize{10}{10}\selectfont{\global\setmainfont{Arial}{\ }}}\textcolor[HTML]{000000}{\fontsize{10}{10}\selectfont{\global\setmainfont{Arial}{dentistry}}}\textcolor[HTML]{000000}{\fontsize{10}{10}\selectfont{\global\setmainfont{Arial}{\ }}}\textcolor[HTML]{000000}{\fontsize{10}{10}\selectfont{\global\setmainfont{Arial}{clincs}}}\textcolor[HTML]{000000}{\fontsize{10}{10}\selectfont{\global\setmainfont{Arial}{\ }}}\textcolor[HTML]{000000}{\fontsize{10}{10}\selectfont{\global\setmainfont{Arial}{were}}}\textcolor[HTML]{000000}{\fontsize{10}{10}\selectfont{\global\setmainfont{Arial}{\ }}}\textcolor[HTML]{000000}{\fontsize{10}{10}\selectfont{\global\setmainfont{Arial}{manually}}}\textcolor[HTML]{000000}{\fontsize{10}{10}\selectfont{\global\setmainfont{Arial}{\ }}}\textcolor[HTML]{000000}{\fontsize{10}{10}\selectfont{\global\setmainfont{Arial}{scraped}}}\textcolor[HTML]{000000}{\fontsize{10}{10}\selectfont{\global\setmainfont{Arial}{\ }}}\textcolor[HTML]{000000}{\fontsize{10}{10}\selectfont{\global\setmainfont{Arial}{from}}}\textcolor[HTML]{000000}{\fontsize{10}{10}\selectfont{\global\setmainfont{Arial}{\ }}}\textcolor[HTML]{000000}{\fontsize{10}{10}\selectfont{\global\setmainfont{Arial}{a}}}\textcolor[HTML]{000000}{\fontsize{10}{10}\selectfont{\global\setmainfont{Arial}{\ }}}\textcolor[HTML]{000000}{\fontsize{10}{10}\selectfont{\global\setmainfont{Arial}{healthcare}}}\textcolor[HTML]{000000}{\fontsize{10}{10}\selectfont{\global\setmainfont{Arial}{\ }}}\textcolor[HTML]{000000}{\fontsize{10}{10}\selectfont{\global\setmainfont{Arial}{services}}}\textcolor[HTML]{000000}{\fontsize{10}{10}\selectfont{\global\setmainfont{Arial}{\ }}}\textcolor[HTML]{000000}{\fontsize{10}{10}\selectfont{\global\setmainfont{Arial}{database}}}\textcolor[HTML]{000000}{\fontsize{10}{10}\selectfont{\global\setmainfont{Arial}{\ }}}\textcolor[HTML]{000000}{\fontsize{10}{10}\selectfont{\global\setmainfont{Arial}{and}}}\textcolor[HTML]{000000}{\fontsize{10}{10}\selectfont{\global\setmainfont{Arial}{\ }}}\textcolor[HTML]{000000}{\fontsize{10}{10}\selectfont{\global\setmainfont{Arial}{checked}}}\textcolor[HTML]{000000}{\fontsize{10}{10}\selectfont{\global\setmainfont{Arial}{\ }}}\textcolor[HTML]{000000}{\fontsize{10}{10}\selectfont{\global\setmainfont{Arial}{via}}}\textcolor[HTML]{000000}{\fontsize{10}{10}\selectfont{\global\setmainfont{Arial}{\ }}}\textcolor[HTML]{000000}{\fontsize{10}{10}\selectfont{\global\setmainfont{Arial}{Google}}}\textcolor[HTML]{000000}{\fontsize{10}{10}\selectfont{\global\setmainfont{Arial}{\ }}}\textcolor[HTML]{000000}{\fontsize{10}{10}\selectfont{\global\setmainfont{Arial}{Maps}}}\textcolor[HTML]{000000}{\fontsize{10}{10}\selectfont{\global\setmainfont{Arial}{\ }}}\textcolor[HTML]{000000}{\fontsize{10}{10}\selectfont{\global\setmainfont{Arial}{to}}}\textcolor[HTML]{000000}{\fontsize{10}{10}\selectfont{\global\setmainfont{Arial}{\ }}}\textcolor[HTML]{000000}{\fontsize{10}{10}\selectfont{\global\setmainfont{Arial}{remove}}}\textcolor[HTML]{000000}{\fontsize{10}{10}\selectfont{\global\setmainfont{Arial}{\ }}}\textcolor[HTML]{000000}{\fontsize{10}{10}\selectfont{\global\setmainfont{Arial}{non-operational}}}\textcolor[HTML]{000000}{\fontsize{10}{10}\selectfont{\global\setmainfont{Arial}{\ }}}\textcolor[HTML]{000000}{\fontsize{10}{10}\selectfont{\global\setmainfont{Arial}{locations}}}\textcolor[HTML]{000000}{\fontsize{10}{10}\selectfont{\global\setmainfont{Arial}{\ }}}\textcolor[HTML]{000000}{\fontsize{10}{10}\selectfont{\global\setmainfont{Arial}{and}}}\textcolor[HTML]{000000}{\fontsize{10}{10}\selectfont{\global\setmainfont{Arial}{\ }}}\textcolor[HTML]{000000}{\fontsize{10}{10}\selectfont{\global\setmainfont{Arial}{confirm}}}\textcolor[HTML]{000000}{\fontsize{10}{10}\selectfont{\global\setmainfont{Arial}{\ }}}\textcolor[HTML]{000000}{\fontsize{10}{10}\selectfont{\global\setmainfont{Arial}{dentistry-orientation.}}}} \\





\multicolumn{1}{>{\raggedright}m{\dimexpr 0.67in+0\tabcolsep}}{\textcolor[HTML]{000000}{\fontsize{10}{10}\selectfont{\global\setmainfont{Arial}{Errand-centric}}}} & \multicolumn{1}{>{\raggedright}m{\dimexpr 1.69in+0\tabcolsep}}{\textcolor[HTML]{000000}{\fontsize{10}{10}\selectfont{\global\setmainfont{Arial}{Hamilton}}}\textcolor[HTML]{000000}{\fontsize{10}{10}\selectfont{\global\setmainfont{Arial}{\ }}}\textcolor[HTML]{000000}{\fontsize{10}{10}\selectfont{\global\setmainfont{Arial}{libraries}}}\textcolor[HTML]{000000}{\fontsize{10}{10}\selectfont{\global\setmainfont{Arial}{\ }}}\textcolor[HTML]{000000}{\fontsize{10}{10}\selectfont{\global\setmainfont{Arial}{(Hamilton}}}\textcolor[HTML]{000000}{\fontsize{10}{10}\selectfont{\global\setmainfont{Arial}{\ }}}\textcolor[HTML]{000000}{\fontsize{10}{10}\selectfont{\global\setmainfont{Arial}{2022b)}}}\textcolor[HTML]{000000}{\fontsize{10}{10}\selectfont{\global\setmainfont{Arial}{,}}}\textcolor[HTML]{000000}{\fontsize{10}{10}\selectfont{\global\setmainfont{Arial}{\ }}}\textcolor[HTML]{000000}{\fontsize{10}{10}\selectfont{\global\setmainfont{Arial}{post}}}\textcolor[HTML]{000000}{\fontsize{10}{10}\selectfont{\global\setmainfont{Arial}{\ }}}\textcolor[HTML]{000000}{\fontsize{10}{10}\selectfont{\global\setmainfont{Arial}{office}}}\textcolor[HTML]{000000}{\fontsize{10}{10}\selectfont{\global\setmainfont{Arial}{\ }}}\textcolor[HTML]{000000}{\fontsize{10}{10}\selectfont{\global\setmainfont{Arial}{locations}}}\textcolor[HTML]{000000}{\fontsize{10}{10}\selectfont{\global\setmainfont{Arial}{\ }}}\textcolor[HTML]{000000}{\fontsize{10}{10}\selectfont{\global\setmainfont{Arial}{(Axle Data}}}\textcolor[HTML]{000000}{\fontsize{10}{10}\selectfont{\global\setmainfont{Arial}{\ }}}\textcolor[HTML]{000000}{\fontsize{10}{10}\selectfont{\global\setmainfont{Arial}{2023;}}}\textcolor[HTML]{000000}{\fontsize{10}{10}\selectfont{\global\setmainfont{Arial}{\ }}}\textcolor[HTML]{000000}{\fontsize{10}{10}\selectfont{\global\setmainfont{Arial}{Canada Post}}}\textcolor[HTML]{000000}{\fontsize{10}{10}\selectfont{\global\setmainfont{Arial}{\ }}}\textcolor[HTML]{000000}{\fontsize{10}{10}\selectfont{\global\setmainfont{Arial}{2023)}}}\textcolor[HTML]{000000}{\fontsize{10}{10}\selectfont{\global\setmainfont{Arial}{,}}}\textcolor[HTML]{000000}{\fontsize{10}{10}\selectfont{\global\setmainfont{Arial}{\ }}}\textcolor[HTML]{000000}{\fontsize{10}{10}\selectfont{\global\setmainfont{Arial}{and}}}\textcolor[HTML]{000000}{\fontsize{10}{10}\selectfont{\global\setmainfont{Arial}{\ }}}\textcolor[HTML]{000000}{\fontsize{10}{10}\selectfont{\global\setmainfont{Arial}{datasets}}}\textcolor[HTML]{000000}{\fontsize{10}{10}\selectfont{\global\setmainfont{Arial}{\ }}}\textcolor[HTML]{000000}{\fontsize{10}{10}\selectfont{\global\setmainfont{Arial}{of}}}\textcolor[HTML]{000000}{\fontsize{10}{10}\selectfont{\global\setmainfont{Arial}{\ }}}\textcolor[HTML]{000000}{\fontsize{10}{10}\selectfont{\global\setmainfont{Arial}{all}}}\textcolor[HTML]{000000}{\fontsize{10}{10}\selectfont{\global\setmainfont{Arial}{\ }}}\textcolor[HTML]{000000}{\fontsize{10}{10}\selectfont{\global\setmainfont{Arial}{national}}}\textcolor[HTML]{000000}{\fontsize{10}{10}\selectfont{\global\setmainfont{Arial}{\ }}}\textcolor[HTML]{000000}{\fontsize{10}{10}\selectfont{\global\setmainfont{Arial}{bank}}}\textcolor[HTML]{000000}{\fontsize{10}{10}\selectfont{\global\setmainfont{Arial}{\ }}}\textcolor[HTML]{000000}{\fontsize{10}{10}\selectfont{\global\setmainfont{Arial}{chains}}}\textcolor[HTML]{000000}{\fontsize{10}{10}\selectfont{\global\setmainfont{Arial}{\ }}}\textcolor[HTML]{000000}{\fontsize{10}{10}\selectfont{\global\setmainfont{Arial}{(BMO}}}\textcolor[HTML]{000000}{\fontsize{10}{10}\selectfont{\global\setmainfont{Arial}{\ }}}\textcolor[HTML]{000000}{\fontsize{10}{10}\selectfont{\global\setmainfont{Arial}{2023;}}}\textcolor[HTML]{000000}{\fontsize{10}{10}\selectfont{\global\setmainfont{Arial}{\ }}}\textcolor[HTML]{000000}{\fontsize{10}{10}\selectfont{\global\setmainfont{Arial}{HSBC}}}\textcolor[HTML]{000000}{\fontsize{10}{10}\selectfont{\global\setmainfont{Arial}{\ }}}\textcolor[HTML]{000000}{\fontsize{10}{10}\selectfont{\global\setmainfont{Arial}{2023;}}}\textcolor[HTML]{000000}{\fontsize{10}{10}\selectfont{\global\setmainfont{Arial}{\ }}}\textcolor[HTML]{000000}{\fontsize{10}{10}\selectfont{\global\setmainfont{Arial}{National Bank}}}\textcolor[HTML]{000000}{\fontsize{10}{10}\selectfont{\global\setmainfont{Arial}{\ }}}\textcolor[HTML]{000000}{\fontsize{10}{10}\selectfont{\global\setmainfont{Arial}{2023;}}}\textcolor[HTML]{000000}{\fontsize{10}{10}\selectfont{\global\setmainfont{Arial}{\ }}}\textcolor[HTML]{000000}{\fontsize{10}{10}\selectfont{\global\setmainfont{Arial}{RBC}}}\textcolor[HTML]{000000}{\fontsize{10}{10}\selectfont{\global\setmainfont{Arial}{\ }}}\textcolor[HTML]{000000}{\fontsize{10}{10}\selectfont{\global\setmainfont{Arial}{2023;}}}\textcolor[HTML]{000000}{\fontsize{10}{10}\selectfont{\global\setmainfont{Arial}{\ }}}\textcolor[HTML]{000000}{\fontsize{10}{10}\selectfont{\global\setmainfont{Arial}{Scotiabank}}}\textcolor[HTML]{000000}{\fontsize{10}{10}\selectfont{\global\setmainfont{Arial}{\ }}}\textcolor[HTML]{000000}{\fontsize{10}{10}\selectfont{\global\setmainfont{Arial}{2023;}}}\textcolor[HTML]{000000}{\fontsize{10}{10}\selectfont{\global\setmainfont{Arial}{\ }}}\textcolor[HTML]{000000}{\fontsize{10}{10}\selectfont{\global\setmainfont{Arial}{TD Bank}}}\textcolor[HTML]{000000}{\fontsize{10}{10}\selectfont{\global\setmainfont{Arial}{\ }}}\textcolor[HTML]{000000}{\fontsize{10}{10}\selectfont{\global\setmainfont{Arial}{2023)}}}\textcolor[HTML]{000000}{\fontsize{10}{10}\selectfont{\global\setmainfont{Arial}{.}}}} & \multicolumn{1}{>{\raggedright}m{\dimexpr 4.33in+0\tabcolsep}}{\textcolor[HTML]{000000}{\fontsize{10}{10}\selectfont{\global\setmainfont{Arial}{Libraries,}}}\textcolor[HTML]{000000}{\fontsize{10}{10}\selectfont{\global\setmainfont{Arial}{\ }}}\textcolor[HTML]{000000}{\fontsize{10}{10}\selectfont{\global\setmainfont{Arial}{post}}}\textcolor[HTML]{000000}{\fontsize{10}{10}\selectfont{\global\setmainfont{Arial}{\ }}}\textcolor[HTML]{000000}{\fontsize{10}{10}\selectfont{\global\setmainfont{Arial}{offices,}}}\textcolor[HTML]{000000}{\fontsize{10}{10}\selectfont{\global\setmainfont{Arial}{\ }}}\textcolor[HTML]{000000}{\fontsize{10}{10}\selectfont{\global\setmainfont{Arial}{and}}}\textcolor[HTML]{000000}{\fontsize{10}{10}\selectfont{\global\setmainfont{Arial}{\ }}}\textcolor[HTML]{000000}{\fontsize{10}{10}\selectfont{\global\setmainfont{Arial}{banks:}}}\textcolor[HTML]{000000}{\fontsize{10}{10}\selectfont{\global\setmainfont{Arial}{\ }}}\textcolor[HTML]{000000}{\fontsize{10}{10}\selectfont{\global\setmainfont{Arial}{158}}}\textcolor[HTML]{000000}{\fontsize{10}{10}\selectfont{\global\setmainfont{Arial}{\ }}}\textcolor[HTML]{000000}{\fontsize{10}{10}\selectfont{\global\setmainfont{Arial}{destinations}}}\textcolor[HTML]{000000}{\fontsize{10}{10}\selectfont{\global\setmainfont{Arial}{\ }}}\textcolor[HTML]{000000}{\fontsize{10}{10}\selectfont{\global\setmainfont{Arial}{are}}}\textcolor[HTML]{000000}{\fontsize{10}{10}\selectfont{\global\setmainfont{Arial}{\ }}}\textcolor[HTML]{000000}{\fontsize{10}{10}\selectfont{\global\setmainfont{Arial}{identified.}}}\textcolor[HTML]{000000}{\fontsize{10}{10}\selectfont{\global\setmainfont{Arial}{\ }}}\textcolor[HTML]{000000}{\fontsize{10}{10}\selectfont{\global\setmainfont{Arial}{Post}}}\textcolor[HTML]{000000}{\fontsize{10}{10}\selectfont{\global\setmainfont{Arial}{\ }}}\textcolor[HTML]{000000}{\fontsize{10}{10}\selectfont{\global\setmainfont{Arial}{offices}}}\textcolor[HTML]{000000}{\fontsize{10}{10}\selectfont{\global\setmainfont{Arial}{\ }}}\textcolor[HTML]{000000}{\fontsize{10}{10}\selectfont{\global\setmainfont{Arial}{are}}}\textcolor[HTML]{000000}{\fontsize{10}{10}\selectfont{\global\setmainfont{Arial}{\ }}}\textcolor[HTML]{000000}{\fontsize{10}{10}\selectfont{\global\setmainfont{Arial}{retrieved}}}\textcolor[HTML]{000000}{\fontsize{10}{10}\selectfont{\global\setmainfont{Arial}{\ }}}\textcolor[HTML]{000000}{\fontsize{10}{10}\selectfont{\global\setmainfont{Arial}{from}}}\textcolor[HTML]{000000}{\fontsize{10}{10}\selectfont{\global\setmainfont{Arial}{\ }}}\textcolor[HTML]{000000}{\fontsize{10}{10}\selectfont{\global\setmainfont{Arial}{a}}}\textcolor[HTML]{000000}{\fontsize{10}{10}\selectfont{\global\setmainfont{Arial}{\ }}}\textcolor[HTML]{000000}{\fontsize{10}{10}\selectfont{\global\setmainfont{Arial}{mix}}}\textcolor[HTML]{000000}{\fontsize{10}{10}\selectfont{\global\setmainfont{Arial}{\ }}}\textcolor[HTML]{000000}{\fontsize{10}{10}\selectfont{\global\setmainfont{Arial}{of}}}\textcolor[HTML]{000000}{\fontsize{10}{10}\selectfont{\global\setmainfont{Arial}{\ }}}\textcolor[HTML]{000000}{\fontsize{10}{10}\selectfont{\global\setmainfont{Arial}{databases,}}}\textcolor[HTML]{000000}{\fontsize{10}{10}\selectfont{\global\setmainfont{Arial}{\ }}}\textcolor[HTML]{000000}{\fontsize{10}{10}\selectfont{\global\setmainfont{Arial}{and}}}\textcolor[HTML]{000000}{\fontsize{10}{10}\selectfont{\global\setmainfont{Arial}{\ }}}\textcolor[HTML]{000000}{\fontsize{10}{10}\selectfont{\global\setmainfont{Arial}{duplicates}}}\textcolor[HTML]{000000}{\fontsize{10}{10}\selectfont{\global\setmainfont{Arial}{\ }}}\textcolor[HTML]{000000}{\fontsize{10}{10}\selectfont{\global\setmainfont{Arial}{are}}}\textcolor[HTML]{000000}{\fontsize{10}{10}\selectfont{\global\setmainfont{Arial}{\ }}}\textcolor[HTML]{000000}{\fontsize{10}{10}\selectfont{\global\setmainfont{Arial}{removed.}}}\textcolor[HTML]{000000}{\fontsize{10}{10}\selectfont{\global\setmainfont{Arial}{\ }}}\textcolor[HTML]{000000}{\fontsize{10}{10}\selectfont{\global\setmainfont{Arial}{Banks}}}\textcolor[HTML]{000000}{\fontsize{10}{10}\selectfont{\global\setmainfont{Arial}{\ }}}\textcolor[HTML]{000000}{\fontsize{10}{10}\selectfont{\global\setmainfont{Arial}{are}}}\textcolor[HTML]{000000}{\fontsize{10}{10}\selectfont{\global\setmainfont{Arial}{\ }}}\textcolor[HTML]{000000}{\fontsize{10}{10}\selectfont{\global\setmainfont{Arial}{also}}}\textcolor[HTML]{000000}{\fontsize{10}{10}\selectfont{\global\setmainfont{Arial}{\ }}}\textcolor[HTML]{000000}{\fontsize{10}{10}\selectfont{\global\setmainfont{Arial}{derived}}}\textcolor[HTML]{000000}{\fontsize{10}{10}\selectfont{\global\setmainfont{Arial}{\ }}}\textcolor[HTML]{000000}{\fontsize{10}{10}\selectfont{\global\setmainfont{Arial}{from}}}\textcolor[HTML]{000000}{\fontsize{10}{10}\selectfont{\global\setmainfont{Arial}{\ }}}\textcolor[HTML]{000000}{\fontsize{10}{10}\selectfont{\global\setmainfont{Arial}{Data}}}\textcolor[HTML]{000000}{\fontsize{10}{10}\selectfont{\global\setmainfont{Arial}{\ }}}\textcolor[HTML]{000000}{\fontsize{10}{10}\selectfont{\global\setmainfont{Arial}{Axle}}}\textcolor[HTML]{000000}{\fontsize{10}{10}\selectfont{\global\setmainfont{Arial}{\ }}}\textcolor[HTML]{000000}{\fontsize{10}{10}\selectfont{\global\setmainfont{Arial}{and}}}\textcolor[HTML]{000000}{\fontsize{10}{10}\selectfont{\global\setmainfont{Arial}{\ }}}\textcolor[HTML]{000000}{\fontsize{10}{10}\selectfont{\global\setmainfont{Arial}{then}}}\textcolor[HTML]{000000}{\fontsize{10}{10}\selectfont{\global\setmainfont{Arial}{\ }}}\textcolor[HTML]{000000}{\fontsize{10}{10}\selectfont{\global\setmainfont{Arial}{cross-referenced}}}\textcolor[HTML]{000000}{\fontsize{10}{10}\selectfont{\global\setmainfont{Arial}{\ }}}\textcolor[HTML]{000000}{\fontsize{10}{10}\selectfont{\global\setmainfont{Arial}{to}}}\textcolor[HTML]{000000}{\fontsize{10}{10}\selectfont{\global\setmainfont{Arial}{\ }}}\textcolor[HTML]{000000}{\fontsize{10}{10}\selectfont{\global\setmainfont{Arial}{ensure}}}\textcolor[HTML]{000000}{\fontsize{10}{10}\selectfont{\global\setmainfont{Arial}{\ }}}\textcolor[HTML]{000000}{\fontsize{10}{10}\selectfont{\global\setmainfont{Arial}{data}}}\textcolor[HTML]{000000}{\fontsize{10}{10}\selectfont{\global\setmainfont{Arial}{\ }}}\textcolor[HTML]{000000}{\fontsize{10}{10}\selectfont{\global\setmainfont{Arial}{quality}}}\textcolor[HTML]{000000}{\fontsize{10}{10}\selectfont{\global\setmainfont{Arial}{\ }}}\textcolor[HTML]{000000}{\fontsize{10}{10}\selectfont{\global\setmainfont{Arial}{with}}}\textcolor[HTML]{000000}{\fontsize{10}{10}\selectfont{\global\setmainfont{Arial}{\ }}}\textcolor[HTML]{000000}{\fontsize{10}{10}\selectfont{\global\setmainfont{Arial}{a}}}\textcolor[HTML]{000000}{\fontsize{10}{10}\selectfont{\global\setmainfont{Arial}{\ }}}\textcolor[HTML]{000000}{\fontsize{10}{10}\selectfont{\global\setmainfont{Arial}{Bank}}}\textcolor[HTML]{000000}{\fontsize{10}{10}\selectfont{\global\setmainfont{Arial}{\ }}}\textcolor[HTML]{000000}{\fontsize{10}{10}\selectfont{\global\setmainfont{Arial}{Locator}}}\textcolor[HTML]{000000}{\fontsize{10}{10}\selectfont{\global\setmainfont{Arial}{\ }}}\textcolor[HTML]{000000}{\fontsize{10}{10}\selectfont{\global\setmainfont{Arial}{website}}}\textcolor[HTML]{000000}{\fontsize{10}{10}\selectfont{\global\setmainfont{Arial}{\ }}}\textcolor[HTML]{000000}{\fontsize{10}{10}\selectfont{\global\setmainfont{Arial}{for}}}\textcolor[HTML]{000000}{\fontsize{10}{10}\selectfont{\global\setmainfont{Arial}{\ }}}\textcolor[HTML]{000000}{\fontsize{10}{10}\selectfont{\global\setmainfont{Arial}{all}}}\textcolor[HTML]{000000}{\fontsize{10}{10}\selectfont{\global\setmainfont{Arial}{\ }}}\textcolor[HTML]{000000}{\fontsize{10}{10}\selectfont{\global\setmainfont{Arial}{national}}}\textcolor[HTML]{000000}{\fontsize{10}{10}\selectfont{\global\setmainfont{Arial}{\ }}}\textcolor[HTML]{000000}{\fontsize{10}{10}\selectfont{\global\setmainfont{Arial}{banking}}}\textcolor[HTML]{000000}{\fontsize{10}{10}\selectfont{\global\setmainfont{Arial}{\ }}}\textcolor[HTML]{000000}{\fontsize{10}{10}\selectfont{\global\setmainfont{Arial}{firms.}}}} \\

\ascline{1.5pt}{666666}{1-3}


\end{longtable}

\arrayrulecolor[HTML]{000000}

\global\setlength{\arrayrulewidth}{\Oldarrayrulewidth}

\global\setlength{\tabcolsep}{\Oldtabcolsep}

\renewcommand*{\arraystretch}{1}

\begin{figure}

\centering{

\includegraphics[width=6.25in,height=\textheight]{figures/Fig2-plot_care_categories.png}

}

\caption{\label{fig-Fig2}The geo-located points of care destinations in
the City of Hamilton separated by the author-generated categories of:
child-, elder-, errand-, grocery- and health- centric care categories.
Locations of these destinations were retrieved through multiple sources
as described in the text. Basemap shapefiles are sourced from the Open
Data Hamilton Portal \citep{opendatahamiltonCityBoundary2023} and the
USGS \citep{greatlakesUSGS2010}.}

\end{figure}%

\subsection{Population data}\label{population-data}

To supplement the care destination dataset and complete the
accessibility calculation (discussed in the followng section),
population data for the City of Hamilton is sourced from the 2021
Canadian census using the \{cancensus\} R Package
\citep{governmentofcanadaCensusPopulation2023, vonbergmannCancensusCensusMapper2021}.
Three categories of variables are selected: the population, the percent
of after-tax low-income-cut-off (LICO), and the primary commute mode
used. LICO is a composite indicator included in the census that reflects
the proportion of households spending 20\% more than the area average on
food, shelter and clothing
\citep{governmentofcanadaLowIncomeCutoffs2023}. As stated in the
Introduction, women, especially those in low-income households, preform
the majority of care trips. However, since the proportion of women and
men residing across the city is balanced, this study focuses on the
total population and total LICO prevalence. All data was sourced at the
most granual level of spatial resolution publicly available, the level
of the dissemination area (DA).

Figure~\ref{fig-Fig3} displays the spatial distribution of the total
population and LICO prevalence as a percentage of the total population.
Notably, the density of population within Hamilton-Central (oranges) and
the cluster of high density and high LICO prevalence near the shoreline
in Hamilton-Central (dark purple-oranges).

\begin{figure}

\centering{

\includegraphics[width=6.25in,height=\textheight]{figures/Fig3-plot_pops.png}

}

\caption{\label{fig-Fig3}The total population in each dissemination area
(DA) as provided in the 2021 Canadian census
\citep{governmentofcanadaCensusPopulation2023}, visualized within the
six community boundaries in the city of Hamilton. The left plot
represents the population and the right represents the population
density versus the low-income cutt-off after taxes (LICO) as a
percentage of the total DA population. LICO is a measure of economic
disadvantage. The legend categories represent quartiles. Basemap
shapefiles are retrieved from the 2021 Canadian census
\citep{governmentofcanadaCensusPopulation2023}, the Open Data Hamilton
Portal \citep{opendatahamiltonCityBoundary2023} and the USGS
\citep{greatlakesUSGS2010}.}

\end{figure}%

Further, the population proportion that commutes by a specific mode
(car, transit, walk, or cycle/other) is visualised in
Figure~\ref{fig-Fig4}. Though mode-choice used in travel to work is not
necessarily reflective of the mode used to travel to care destinations,
no other data is available at a granular level City-wide that centers
mobility of care to our knowledge. The population generally commutes by
car (50\% or higher, is yellow to green), even within the more densely
populated Hamilton-Central. However, for transit and walking, a group of
DAs near the shoreline within Hamilton-Central have the highest
proportion of transit users and those who walk to work (yellows in the
plots that are otherwise red i.e., below 15\%). Those same DAs are also
relatively dense and have a high prevalence of LICO
(Figure~\ref{fig-Fig3}).

\begin{figure}

\centering{

\includegraphics[width=6.25in,height=\textheight]{figures/Fig4-plot_modal_splits.png}

}

\caption{\label{fig-Fig4}The proportion of mode type used for commuting
(aged 15 and older employed in the labour force) in each dissemination
area (DA) as provided by the 2021 Canadian census
\citep{governmentofcanadaCensusPopulation2023}. Basemap shapefiles are
retrieved from the 2021 Canadian census
\citep{governmentofcanadaCensusPopulation2023}, the Open Data Hamilton
Portal \citep{opendatahamiltonCityBoundary2023} and the USGS
\citep{greatlakesUSGS2010}.}

\end{figure}%

\subsection{Transportation network and travel time
estimations}\label{transportation-network-and-travel-time-estimations}

As empirical travel behaviour to care-oriented destinations is uncounted
and thus travel time is unavailable, travel time is approximated. Travel
times by walking, cycling, transit and car is calculated for the
geometric centroids of the DAs to the geometric centroids of the care
destination location using the `travel\_time\_matrix()' function from
the \{r5r\} package \citep{pereiraR5rRapidRealistic2021}. Inputs are
point locations of DA centroids (origins), care destinations centroids,
an OpenStreetMap road network including bike, transit and vehicle
infrastructure \citep{geofabrikOntarioCanadaOpen2023}, and city GTFS
transit routes/schedules \citep{transitfeedsHamiltonStreetRailway2023}.
For all modes, travel times under 60 minutes based on the shortest
travel-time path are calculated.

For transit and cycling, additional parameters were included. For
transit travel times, a Wednesday departure time of 8:00AM was selected
\citep{boisjolyDailyFluctuationsTransit2016} with a departure travel
window parameter of 30 mins. Travel times are calculated for each minute
of the travel window (8:00-8:30AM) and the 25th percentile from the
distribution of travel window times were selected to represent each
origin-destination. Selecting a sufficiently wide window is an important
consideration as travel times are sensitive to transit vehicle frequency
and connecting transfers (see discussion of the modifiable temporal unit
problem e.g., \citep{pereiraFutureAccessibilityImpacts2019}). The 25th
percentile indicates that 25\% of trips from that origin to destination
have a travel time that is that length or shorter. This assumption
provides an optimistic perspective of transit travel times. For cycling
travel times, level 1 or 2 traffic level of stress routes (i.e.,
dedicated or separated cycling lanes respectively) were selected. The
level of traffic stress is a variable associated with links of the OSM
road network; level 1 and 2 are considered the default.

\section{Accessibility measurement
methods}\label{accessibility-measurement-methods}

Two accessibility measures are detailed: the cumulative opportunities
measure and the spatial availability measure. Both yield a value per
spatial unit that represents how many care destinations can be reached
within a given travel time, for a given mode. However, both measures
have different underlying assumptions; the first does not consider
competition effects and the second does.

\subsection{Cumulative opportunities: the number of care opportunities
that can be reached by a mode within a travel
time}\label{cumulative-opportunities-the-number-of-care-opportunities-that-can-be-reached-by-a-mode-within-a-travel-time}

Often referred to as the cumulative opportunities measure, it is a
special form of the gravity-based accessibility measure used in at least
as far back as \citeauthor{hansenHowAccessibilityShapes1959}
\citetext{\citeyear{hansenHowAccessibilityShapes1959}; \citealp{handyMeasuringAccessibilityExploration1997}}.
Its name is drawn from its' interpretation: the value calculated for
each spatial unit (DAs in this study) represents the number of
opportunities that can be spatially accessed within a given travel time.
The cumulative opportunities accessibility measure takes the following
general form for a multimodal calculation: \[
S_i^m=\sum_{j}O_j\cdot f^m(c_{ij}^m)
\] \noindent Where:

\begin{itemize}
\tightlist
\item
  \(i\) is a set of origin locations (e.g., DA centroids)
\item
  \(j\) is a set of destination locations (e.g., care destinations)
\item
  \(m\) is a set of modes (e.g., by foot, cycle, transit and car)
\item
  \(O_j\) is the number of opportunities at \(j\) (e.g., the presence of
  a care destination in this study)
\item
  \(c_{ij}^m\) is the travel cost between \(i\) and \(j\) for each
  \(m\).
\item
  \(f^m(\cdot)\) is an impedance function of \(c^m_{ij}\) for each
  \(m\); within the cumulative opportunities measure, it is a binary
  function that takes the value of 1 if \(c^m_{ij}\) is less than a
  selected value.
\item
  \(S^m_{i}\) is the cumulative opportunities accessible by \(m\) at
  each \(i\).
\end{itemize}

\subsection{Spatial availability: the number of care opportunities that
are spatially available to a mode-user within a travel
time}\label{spatial-availability-the-number-of-care-opportunities-that-are-spatially-available-to-a-mode-user-within-a-travel-time}

Differing from cumulative opportunities measure, the spatial
availability measure considers competition leading to a different
interpretation in its results. The values for each origin \(i\) (in our
study, DAs) for a given mode \(m\) represents the number of care
opportunities that can be accessed by a mode-user out of \emph{all} care
opportunities in Hamilton. Spatial availability considers competition
through proportional allocation of opportunities to a given \(i\) based
on the relative proportion of population computing for an opportunity
and their travel times. Each \(V_i\) value represents the potential
availability of reachable destinations. Spatial availability, takes the
following general form for multimodal calculation: \[
V^m_{i} = \sum_{j} O_j\ F^{tm}_{ij}
\] \noindent Where:

\begin{itemize}
\tightlist
\item
  Like in Equation (1), \(i\), \(j\), and \(m\) is a set of origin
  locations, destination locations, modes respectively and \(O_j\) is
  the number of opportunities at \(j\).
\item
  \(V^m_{i}\) is the cumulative opportunities spatially available by
  \(m\)-using population at \(i\) for each \(i\).
\item
  \(F^{tm}_{ij}\) is a total balancing factor for each \(m\) at each
  \(i\); it considers the size of the populations at different locations
  that demand opportunities \(O_j\), as well as the cost of movement in
  the system \(f(c_{ij})\).
\end{itemize}

What makes spatial availability stand apart from other competitive
measures is the multimodal balancing factor \(F^{tm}_{ij}\)
\citep{soukhovMultimodalSpatialAvailability2024, soukhovIntroducingSpatialAvailability2023}.
\(F^{tm}_{ij}\) implements a proportional allocation mechanism that
ensures the sum of all spatial availability values \(V^m_{i}\) across
all modes \(m\) in the region always matches the total number of
opportunities (i.e.,
\(\sum_j O_j = \sum_i V_i = \sum_{m} \sum_{i} V^m_{i}\)). This
constraint helps in clarifying the interpretation of the \(V^m_{i}\)
value itself.

The total proportional allocation factor \(F^{tm}_{ij}\) consists of two
parts: the first is a population-based proportional allocation factor
\(F_i^{pm}\) that models the mass effect (relative population-demand for
opportunities) and the second is an impedance-based proportional
allocation factor \(F_{ij}^{cm}\) that models the cost effect (relative
travel time). Both factors consider competition through proportional
allocation: \(F^{pm}_{i}\) estimates a proportion of how many people are
in each \(i\) and using each \(m\) relative to the region and
\(F^{cm}_{ij}\) estimates a proportion of the cost of travel from \(i\)
to \(j\) at each \(i\) using each \(m\) relative to the region. Since
\(F^{pm}_{i}\) and \(F^{mc}_{i}\) are proportions,
\(\sum_{m}\sum_{i}F^{pm}_{i} = 1\) and
\(\sum_{m} \sum_{i}F^{cm}_{i}=1\). Both factors are combined to create
the total balancing factor \(F^{tm}_{ij}\) used to calculate \(V^m_i\):

\[
F^{tm}_{ij} = \frac{F^{pm}_{i} \cdot F^{cm}_{ij}}{\sum_{m} \sum_{i} F^{pm}_{i} \cdot F^{cm}_{ij}}
\] \noindent Where:

\begin{itemize}
\tightlist
\item
  The factor for allocation by population for each \(m\) at each \(i\)
  is \(F^{pm}_{i} = \frac{P_{i}^m}{\sum_{m}\sum_{i} P_{i}^m}\). This
  factor makes opportunities available based on demand.
\item
  The factor for allocation by cost of travel for each \(m\) at \(i\) is
  \(F_{ij}^{cm} = \frac{f^m(c_{ij}^m)}{\sum_{m} \sum_{i} f^m(c_{ij}^m)}\).
  This factor makes opportunities available preferentially to those who
  can reach them at a lower cost.
\end{itemize}

\subsection{Travel impedance function
selection}\label{travel-impedance-function-selection}

A binary travel impedance function \(f^m(c_{ij}^m)\) is assumed (e.g.,
\(c_{ij}\) is equal or below a certain travel time threshold,
\(f^m(c_{ij}^m)\) equals 1, otherwise, \(f^m(c_{ij}^m)\) equals 0). Two
travel time thresholds are selected for both measures: 15 minutes and 30
minutes for all modes.

This selection is informed by a scan of the literature. Typically,
literature considers travel to one type of care category (e.g., health,
or school, or grocery stores) and each destination type is associated
with different travel impedance behaviour (e.g., grocery shopping trips
are on average 15 minutes \citep{hamrickTimeCostAccess2012}, trips to
receive cancer treatments are on average 20 minutes
\citep{segelRuralurbanDifferencesAssociation2020}). In other
care-related accessibility analyses, travel time thresholds of include
10 mins (for daycares) \citep{fransenCommuterbasedTwostepFloating2015}
and 30 mins to 1 hr (for hospitals)
\citep{schuurmanDefiningRationalHospital2006} are selected. Of the one
study to-date that has calculated the average travel times to all
different categories of care destinations, travel times to each care
category differ by mode e.g., 16 minutes by car and 36 minutes by public
transport \citep{ravensbergenExploratoryAnalysisMobility2022}. To
broadly reflect this past research: 15 and 30 minutes are selected for
all modes.

Notably, the use of a binary travel impedance functions as opposed to a
distance-decay impedance function, were selected to simplify
communication of the assumed travel behaviour. As mentioned, lacking
region-specific empirical data regarding care-centric travel, this work
establishes a methodology to streamline access to care interpretation
and analysis for when that data is available.

\section{Results}\label{results}

\subsection{Spatial access to care
opportunities}\label{spatial-access-to-care-opportunities}

The cumulative opportunities and spatial availability plots for each
mode, for both 15-minute and 30-minute travel time thresholds are shown
in Figure~\ref{fig-Fig5}. Each cumulative opportunities value represents
a cumulative count of care opportunities that can be spatially accessed
by each mode from each DA, where each opportunity represents a reachable
care destination. In this case study, the spatial availability measure
presents a constrained interpretation of this measure; each value is a
cumulative count of care opportunities that can be spatially accessed
from each DA \emph{and are spatially available to the mode-using
population based on the relative size of the mode-using population and
modal travel times}. As proportional allocation is used, each spatial
availability value can also be interpreted as the \emph{spatially
available} proportion of the total care destinations in the city, i.e.,
the sum of all spatial availability values in the second row of
Figure~\ref{fig-Fig5} equal 2,225, the total number of care destinations
in this case study.

In both measures, the higher the value, the more potential interaction
with care opportunities. This greater potential of opportunity of
interaction is conceptualised as a positive outcomes of well functioning
land-use and transport networks
\citep{corderaImpactAccessibilityPublic2019, blumenbergDriveWorkRelationship2017, cuiSpatialAccessPublic2020}.
In Figure~\ref{fig-Fig5}, values are grouped by quantile and spatial
trends between the 15-min and 30-min threshold plots are highly
correlated (0.92 for cumulative opportunities and for 0.89spatial
availability).

\begin{figure}

\centering{

\includegraphics[width=6.25in,height=\textheight]{figures/Fig5-plot_copp_Savail_measures.png}

}

\caption{\label{fig-Fig5}The number of care destinations that can be
reached, per DA, within 15 mins (top) and 30 mins (bottom) for the
cumulative opportunities and spatial availability measures. Basemap
shapefiles are retrieved from the 2021 Canadian census
\citep{governmentofcanadaCensusPopulation2023}, the Open Data Hamilton
Portal \citep{opendatahamiltonCityBoundary2023} and the USGS
\citep{greatlakesUSGS2010}.}

\end{figure}%

When considering the cumulative opportunities measure; three notable
findings between modes can be identified. First, access by transit and
walking is somewhat high (mostly Q3 and some Q4) within the core of
Hamilton-Central but low elsewhere. This finding is somewhat expected as
as transit does not significantly serve communities outside of
Hamilton-Central and Dundas and the density of walking infrastructure is
high in Hamilton-Central (see Figure~\ref{fig-Fig1}). Second, access by
cycling is even higher (mostly Q3 but more Q4) in Hamilton-Central; it
provides the second most opportunities for interactions after travel by
car, and affords at least one opportunity for interaction in more DAs
than walking and transit use (notably some access (Q1) in rural
communities). Third, the access that the car-mode provides is
significantly higher relative to the three sustainable modes. Travel by
car results in the greatest maximum number of potential interactions to
care destinations (1818 and 2215 opportunities within 15-min and 30-mins
respectively). Car-mode offering high accessibility to care destinations
is an expected outcome given the car-oriented design of North American
cities \citep{saeidizandRevisitingCarDependency2022} and the range
(travel speeds over a distance) of the car mode. However, though car
ownership is high in Hamilton, not everyone has access to a private
vehicle. For instance, 13\% of Hamilton households own zero vehicles
\citep{datamanagementgroupTTSTransportationTomorrow2018}, presenting
equity concerns in who may benefit from the high accessibility car-mode
offers. The cumulative opportunities access is insightful in
illustrating the range in which opportunities can be accessed by each
mode based on their travel speed (on available infrastructure); a
summary of each origins' modal opportunity isochrone. However, the
cumulative opportunities measure does not account for competition
effects. Namely, what proportion of the modal opportunity range is
\emph{spatially available} to a mode-user at a given location when
competing for those same opportunities with other mode-users.
Considering competition in this way conjures richer conclusions that
reflects the mode-using population. For instance, consider cycling, a
mode that offers a relatively high range but still smaller than the car.
The cumulative opportunities values in Figure~\ref{fig-Fig5} reflects
this intuition: Q3 and Q4 cumulative opportunities values are present
for cycling in Hamilton-Central, offering the second best cumulative
opportunities after the car. However, bike spatial availability values
depicts a more complex story of opportunity accessibility: it reflect
the mode's opportunity range as well as proportion of mode-using
population and how their range relatively compares to all other modes.
The bike-using population is small (2\% of the population), with many
DAs having no or low proportions of bike-users. Meaning DAs with no
bike-users are proportionally allocated no access to opportunities (zero
spatial availability) and DAs with a small proportion of cyclists have
relatively slow travel speeds compared to the car-using population.
Though bike mode offers a relatively high opportunity range (cumulative
opportunities), because of the low proportion of cyclist and their
opportunity range compared to the \emph{many} other mode-users, they
receive low spatial availability values.

In the case study, spatial availability values reflect the proportion of
cumulative opportunities accessibility to the mode user (based on
relative population and travel times), which can be used to shed light
on what mode, and in what region, a mode-using population captures more
than its equal share of spatial availability. Overall, 98\% of the
spatial availability is taken by motorists (destinations within
30-minutes) but they only represent 87\% of the population. Therefore,
they have disproportionately more availability than their population's
presence in the city. Motorists capture this availability from
populations that do not use cars, and as a result are left with lower
spatial availability. For instance, transit users that have access to
destinations within 30-minutes represent 7\% of the population but claim
only 2\% of the spatial availability. Similarly, though cyclists and
pedestrians represent 2\% and 4\%of the population respectively, they
only capture 0.3\% (cyclist) and 0.3\% (pedestrian) of the spatial
availability. In other words, if certain mode-users capture a greater
proportion of spatial availability, then there is less spatial
availability remaining for other mode users. Spatial availability does
not necessarily have to align with the cumulative opportunities that the
mode offers, it is simply a constrained version that considers
competition by mode-using populations. As noted, non-car modes have the
potential to offer higher cumulative opportunities (within
Hamilton-Central), but as it exists assuming modal commute shares, the
majority of spatial availability to care destinations can still be
captured by motorists even in DAs where car mode share is under 50\%
(such as Hamilton-Central, see proportions in Figure~\ref{fig-Fig4}).

Taken together, though non-car modes may provide somewhat good access to
care destinations within Hamilton-Central (and some only some access in
rural communities), they do not provide similar levels of spatial
availability. Car-using populations capture more spatial availability,
even in the centre of Hamilton-Central, than all other modes. Note the
lower number of Q3 and Q4 values within and radiating outwards from
Hamilton-Central for non-car modes for cumulative opportunities measure
compared to spatial availability. This indicates that cumulative
opportunities measures may overestimate the access to care destinations
that non-car modes (pedestrians, cyclists, and transit users) have
available to them.

\subsection{Spatial availability and low-income
mismatch}\label{spatial-availability-and-low-income-mismatch}

To draw insights on who may reside in DAs where populations are
disadvantaged with low modal spatial availability and high low-income
prevalence, a cross-tabulation is visualised in Figure~\ref{fig-Fig6}.
The modal spatial availability is divided by the mode-using population
in each DA, resulting in the rate of modal spatial availability. LICO
prevalence is the proportion of population that falls below the
low-income cuttoff threshold (see Figure~\ref{fig-Fig3}).
Figure~\ref{fig-Fig6} can be interpreted as follows: residents who use a
specific mode in a ``yellow'' DA reside in a DA that offers below
average spatial availability (i.e., below or equal to the the 50th
percentile (median) levels of spatial availability per mode-using
population) and the population within the DA has a high LICO-prevalence
(i.e, 80th percentile or higher (8.4\% or more)).

\begin{figure}

\centering{

\includegraphics[width=6.25in,height=\textheight]{figures/Fig6-plot_Savail_smallv_LICO_measures.png}

}

\caption{\label{fig-Fig6}The spatial availability per mode-using-capita
measure versus LICO prevelance, visualized for 15 mins (top) and 30 mins
(bottom) travel time cutoffs. Basemap shapefiles are retrieved from the
2021 Canadian census \citep{governmentofcanadaCensusPopulation2023}, the
Open Data Hamilton Portal \citep{opendatahamiltonCityBoundary2023} and
the USGS \citep{greatlakesUSGS2010}.}

\end{figure}%

Notice the green DAs for the car-driving population and presence of
yellow DAs for non-car modes within Hamilton-Central:
Figure~\ref{fig-Fig6} reinforces findings from Figure~\ref{fig-Fig5}.
Even in Hamilton-Central where there is high proportion of LICO
prevalence, car-mode using populations who reside in green DAs are still
offered high levels of spatial availability. However, car ownership is
not always possible for low-income households and the lack of ownership
acts as a barrier to accessing economic and economic-support
opportunities for low-income households \citep{morrisDoesLackingCar2020}
when alternative modes are insufficient
\citep{kleinTransitionsOutCar2023}. For this reason, populations below
the LICO may rely on non-car modes, and the introduction of policies
that increase access to care-destinations could be considered. The
majority of yellow DAs are within the centre of Hamilton-Central,
specifically for cycle- and walking populations. Policies that increase
the number of available care-destinations within Hamilton-Central,
improve conditions that decrease LICO-AT prevalence, as well as policies
that make car-modes less spatial availability advantaged (i.e.,
encourage modal shift and decrease travel time) could be further
investigated through the lens of mobility of care.

\section{Discussion and conclusions}\label{discussion-and-conclusions}

This paper is the first to conduct an exploratory multimodal
accessibility analysis of Mobility of Care destinations -- one that
counters the current literature's emphasis on employment-related
destinations, a travel purpose more significant for men, and especially
wealthy and educated men
\citep{lawWomenTransportNew1999, hansonGenderMobilityNew2010}. Its aim
is to challenge current planning paradigms by explicitly focusing on
care destinations, locations that are vital for life-sustaining
activities that are currently undervalued. This study also provides a
tangible example of how one could conduct gender-aware multimodal
accessibility analyses using the City of Hamilton as a case study. In
doing so, this paper contributes to the emergent mobility of care
literature, a body of what that has, to date, focused on quantifying
this under-represented type of travel
\citep{gomezvaroAccountingCareEveryday2023, murillomunarCaregiversMoveGender2023, ravensbergenExploratoryAnalysisMobility2022, sanchezdemadariagaMobilityCareIntroducing2013, sanchezdemadariagaMeasuringMobilitiesCare2019, shumanCanMobilityCare2023}
and providing rich and nuanced qualitative accounts of lived experiences
completing mobility of care
\citep{orjuelaReconsideringMobilityCare2023, ravensbergenVelomobilitiesCareLowcycling2020, sersliRidingAloneTogether2020}.

This study also methodologically contributes to the accessibility
literature by contrasting two multimodal accessibility measures: the
widely used cumulative opportunities measure and the spatial
availability measure, which offers accessibility insights on modal
competition. The cumulative opportunities measure demonstrates the modal
range of access by presenting the number of care destinations that each
mode can reach within a 15- and 30- minute travel time threshold from
each spatial location. Spatial availability constrains the cumulative
opportunities measure by incorporating the \emph{assumed} proportions of
mode-using populations and mode-specific travel times; this yields the
number of care destinations that the mode-using population has access to
out of all care destinations in the study region. The two measures
communicate different insights about the case study: the study's results
demonstrate that the car mode offers high cumulative opportunities
access as well as exceptionally high spatial availability for motorists.
While sustainable modes offer lower cumulative opportunities access
(though higher in the city center) and, in certain areas, even lower
spatial availability due to the disproportionately high spatial
availability for the car users. In this way, relying only on the
cumulative opportunities measure provides an incomplete picture, as it
does not reflect how the relatively large quantity of motorists and the
greater range offered by the car can disproportionately claim more care
destinations than non-car modes (pedestrians, cyclists, and transit)
users. Although spatial availability offers a more complex picture of
how modes provide access under competition, spatial availability like
other competitive measures, relies on assumptions about who is
``demanding'' destinations and by how much. How those assumptions are
made are a subject of ongoing discussion in the competitive
accessibility literature
\citep{merlinDoesCompetitionMatter2017, kelobonyeMeasuringAccessibilitySpatial2020}.

Further, this study contributes to the literature on equitable and
sustainable transportation planning by providing a methodology to
identify areas in need for further development. By highlighting how the
car offers all-round high access and even higher spatial availability to
care destinations in Hamilton, sustainable modes can be prioritized
equitably. Previous research suggests that currently care trips are more
frequently completed by car than by transit or bicycle
\citep{ravensbergenExploratoryAnalysisMobility2022} as they often
involve carrying things (e.g., groceries) or people (e.g., children).
Qualitative work supports this preference, citing convenience and
increased safety as key reasons for choosing travel by car for care
trips
\citep{maciejewskaHaveChildrenThus2019, carverParentalChauffeursWhat2013}.

However, this study also highlights that the high spatial availability
of motorists results in disproportionately low spatial availability for
sustainable mode users, even in Hamilton-Central. While sustainability
policies should aim to re-balance the spatial availability away from
motorists to users of sustainable modes, these policies should
incorporate an equity perspective that considers existing preferences in
care trips. This study provides the stepping stones for such an equity
lens in Figure~\ref{fig-Fig6}, by presenting a cross-tabulation of areas
with high LICO prevalence and low spatial availability per
sustainable-mode that could be the focus of policy intervention.
Consider the cycling plot in Figure~\ref{fig-Fig6}, a factor driving the
higher quantity of yellow DAs is the low proportion of cyclists assumed.
This assumption holds in other Canadian contexts, cycling as a mode for
care trips is also uncommon as cycling is uncommon
\citep{ravensbergenExploratoryAnalysisMobility2022}. Moreover, as care
trips tend to be preformed by women, the low proportion of cycling for
care trips has been put forth as a hypothesis to explain the gender-gap
in cycling observed in low-cycling cities (like Hamilton) where only a
third of cyclists are women
\citep{ravensbergenFeministGeographiesCycling2019, prati2018gender}.
However, cycling as a mode has potential as it demonstrates high
cumulative opportunities values. However, that potential is not being
realized in part due to the low proportion of cyclists and the higher
spatial availability values of motorists. Future research could examine
what barriers those who conduct care trips are facing in regards to
cycling, particularly focusing on the yellow areas indicated
Figure~\ref{fig-Fig6}.

\subsection{Study limitations}\label{study-limitations}

This study presents three types of limitations related to assumptions in
the accessibility measure methods and data availability. First, since
travel times from origin to care destination are unknown, they are
estimated assuming a road network under free-flow conditions. While this
affects the estimated travel times, research suggest that considering
congested conditions may not significantly impact the resulting
accessibility values \citep{yiannakouliasEstimatingEffectTurn2013}. In
the context of Hamilton, congestion is also pertinent to car and transit
modes, and not for pedestrians or cyclists (their travel
infrastructure). Second, using a binary impedance function instead of a
more complex distance-decay function could significantly affect
accessibility results
\citep{kapatsilaResolvingAccessibilityDilemma2023}. For instance,
destinations beyond a 30-minute travel time could still be valued by
people, and those within 5 and 15 minutes do not necessarily have the
same importance. However, the use of the binary impedance function
trades complexity for interpretation, and this trade-off was made
strategically made to improve interpretability and compare the two
accessibility measures. To enhance reliability, two literature-informed
travel cost thresholds (15-minutes and 30-minutes) are selected. Third,
the geometric centroids of DAs (origins) and destinations (all care
destinations) were used as inputs for travel time calculations. This is
a limitation as DAs were created for the purpose of the statistic
census: they vary in area and their centroids may not necessarily align
with where that population may begin their journey to care destinations.
This methodological decision presents limitations on how the travel time
estimates can be interpreted to reflect actual travel times to care
destinations.

Moreover, due to the exploratory nature of this research and novelty of
the Mobility of Care concept, no research to date has directly captured
the characteristics of mobility of care trips in Hamilton. The presented
results thus are not calibrated to reflect observed mobility of care
travel behaviour nor establish normative accessibility goals
\citep{paezMeasuringAccessibilityPositive2012}. Travel behaviour data is
needed to calibrate local destination-specific travel impedance cutoffs
(e.g., using a 15 minute cutoff for grocery-centric destinations and 30
minute cutoff for health-centric destinations) or assigning weights for
each destination type as done in previous studies (e.g., a weight that
reflects their `capacity' \citep{liMeasuringMultiactivities2024} or
their `attractiveness' using origin-destination flows from travel
surveys
\citep{graellsCityCitiesMeasuring2021, chengInvestigatingWalkingAccessibility2019}).
In absence of travel behaviour data, this study uses uniform travel time
thresholds for all destinations and no destination weights are applied.
This limits the result's interpretation to the \emph{potential} to
access \emph{all} care destinations within 15- or 30- minutes, it does
not include the real individual socio-economic and intersectional
characteristics that influence what destinations can be potentially
accessed. Consequently, each destination is treated as a single
opportunity, e.g., a school, a clinic, a hospital, and a grocery store
are all equal to one opportunity each. Additionally, since care trip
modal choice is unavailable at a disaggregated level for Hamilton, the
commute mode choice is assumed for the spatial availability measure.
This mode may not be what is used to visit care destinations and hence
places an limitation on how the results should be interpreted.

Taken together, the discussion of these limitations present room for
future research to incorporate context-specific mobility of care travel
surveys into accessibility analysis to more accurately reflect mobility
of care accessibility landscapes. Future work could also look to
disaggregate access to care by category and compare results to
conventional access to work landscapes. This comparison could highlight
the bias in planning towards jobs as well as substantiate equity
critiques.

\section{References}\label{references}


  \bibliography{bibliography.bib}


\end{document}
